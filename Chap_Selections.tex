\chapter{Sample Selections and Systematics}
\label{chap:SelsAndSysts}

\section{Systematic Uncertainties}
\label{sec:SelsAndSysts_Systs}

The systematics for this uncertainty are split into the groups, or blocks, depending on their purpose. They consist of flux uncertainties, neutrino-matter interaction systematics and detector efficiencies. There are also uncertainties on the oscillation parameters which this analysis will not be sensitive to, \delmsqsol and \sinsqsol. As described in \autoref{chap:MarkovChainMonteCarlo}, each model parameter used within this analysis requires a prior uncertainty. This is provided via separate covariance matrices for each block. The covariance matrices can include prior correlations between parameters within a single block, but the separate treatment means prior uncertainties can not be included for parameters in different groups. Alternatively, some parameters have no reasonably motivated uncertanities. These parameters are assigned flat priors which do not change the likelihood penalty. The flux, neutrino interaction and detector modelling has already been discussed in \autoref{sec:Simulations_Simulation}. The uncertainties invoked within these models are described below.

\subsection{Beam Flux}
\label{sec:SelsAndSysts_Systs_BeamFlux}

The neutrino beam flux systematics are based upon our uncertainty in the modelling of the components of the beam. This includes: the hadron production model and their re-interactions, the shape, intensity and alignment of the beam with respect to the target, and the uniformity of the magnetic field produced by the horn, alongside other effects. The uncertainty, as a function of neutrino energy, is illustrated in \autoref{fig:SelsAndSysts_BeamFluxSysts} which includes the total uncertainty as well as the individual components. The uncertainty for events below, and much higher than, the peak neutrino energy is dominated by hadron production and re-interaction systematics. The beam profile and alignment of the proton beam dominates the systematic uncertainty for events with \quickmath{E_{\nu} \sim 1\text{GeV}}. 

\begin{figure}[h]
  \begin{subfigure}[t]{\textwidth}
    \includegraphics[width=\textwidth, trim={0mm 0mm 0mm 0mm}, clip,page=1]{Figures/Selections/flux_uncertainty_covariance_plots_addcorrnd_compwv3_flux_error_t2k_nd5_fhc_numu.pdf}
  \end{subfigure}
  %\begin{subfigure}[t]{\textwidth}
  %  \includegraphics[width=\textwidth, trim={0mm 0mm 0mm 0mm}, clip,page=1]{Figures/Selections/flux_uncertainty_covariance_plots_addcorrnd_compwv3_flux_error_t2k_sk_fhc_numu.pdf}
  %\end{subfigure}
  \caption{The total uncertainty evaluated on the near detector \quickmath{\nu_{\mu}} flux prediction constrained by the replica-target data, illustrated as a function of neutrino energy. The solid(dashed) line indicates the uncertainty used within this analysis(the T2K 2018 analysis). The solid histogram indicates the neutrino flux as a function of energy. Figure taken from \cite{t2k_tn_354}.}
  \label{fig:SelsAndSysts_BeamFluxSysts}
\end{figure}

The beam flux uncertainties are described by one hundred parameters. They are split between both ND280 and SK detectors and binned by neutrino flavour: \quickmath{\nu_{\mu}}, \quickmath{\bar{\nu}_{\mu}}, \quickmath{\nu_{e}} and \quickmath{\bar{\nu}_{e}}. The response is then broken down as a function of neutrino energy. The bin density in the neutrino energy is the same for the FHC-\quickmath{\nu_{\mu}} and RHC-\quickmath{\bar{\nu}_{\mu}}, and narrows for neutrino energies close to the oscillation maxima of \quickmath{E_{\nu} = 0.6\text{GeV}}. This binning is specified in \autoref{tab:SelsAndSysts_BeamFluxBinEdges}. All of these systematic uncertanties are applied as normalisation parameters with Gaussian priors centered at \quickmath{1.0} and error specified from a covariance matrix provided by the T2K beam group.

\begin{table}[ht!]
    \centering
    \begin{tabular}{c|c|c}
      \hline
      Neutrino Flavour & Sign & Neutrino Energy Bin Edges (GeV) \\
      \hline
      \quickmath{\mu} & Right & \quickmath{0.,0.4,0.5,0.6,0.7,1.,1.5,2.5,3.5,5.,7.,30.} \\
      \quickmath{\mu} & Wrong & \quickmath{0.,0.7,1.,1.5,2.5,30.} \\
      \quickmath{e} & Right & \quickmath{0.,0.5,0.7,0.8,1.5,2.5,4.,30.} \\
      \quickmath{e} & Wrong & \quickmath{0.,2.5,30.} \\
      \hline
      \hline
    \end{tabular}
    \caption{The neutrino energy binning for the different neutrino flavours. ``Right'' sign indicates neutrinos in the FHC beam and antineutrinos in the RHC beam mode. ``Wrong'' sign indicates antineutrinos in the FHC beam and neutrinos in the RHC beam mode. The binning of the detector response is identical for the FHC and RHC modes as well as at ND280 and SK.}
    \label{tab:SelsAndSysts_BeamFluxBinEdges}
\end{table}

\subsection{Atmospheric Flux}
\label{sec:SelsAndSysts_Systs_AtmFlux}
The atmospheric neutrnio flux is modelled by the HKKM model, however 16 systematic uncertainties are applied to control the normalisation of each neutrino flavour, energy and direction. All of the parameters are given Gaussian priors centered at \quickmath{0} and width \quickmath{1.}. They are summarised below:

\begin{itemize}
\item \textbf{Absolute Normalisation}: The overall normalisation of each neutrino flavour is controlled by two indpendent systematic uncertainties, for \quickmath{E_{\nu} < 1\text{GeV}} and \quickmath{E_{\nu} > 1\text{GeV}}, respectively. This is driven mostly by hadronic interaction uncertainties for the production of pions and kaons \cite{Honda_2007}. The strength of the response is dependent upon the neutrino energy.
\item \textbf{Relative Normalisation}: Uncertainties on the ratio of \quickmath{(\nu_{\mu} + \bar{\nu}_{\mu})/(\nu_{e} + \bar{\nu}_{e})} are controlled by the difference between the HKKM model \cite{Honda_2007}, FLUKA \cite{etde_20239111} and Bartol models \cite{Barr_2004}. Three independent parameters are applied in the energy ranges: \quickmath{E_{\nu} < 1\text{GeV}}, \quickmath{1\text{GeV} < E_{\nu} < 10 \text{GeV}}, and \quickmath{E_{\nu} > 10\text{GeV}}.
\item \textbf{\quickmath{\nu}/\quickmath{\bar{\nu}} Normalisation}: The uncertainties in the \quickmath{\pi^{+}/\pi^{-}} (and kaon equivalent) produce uncertainties in the flux of \quickmath{\nu/\bar{\nu}}. The response is applied in the same way as the relative normalisation parameters.
\item \textbf{Up/Down and Vertical/Horizontal Ratio}: Similar to the above two systematics, the difference between the HKKM, FLUKA and Bartol model predictions, as a function of \quickmath{\cos(\theta_{Z})}, is used to control the normalisation of events as a function of zenith angle.
\item{\textbf{\quickmath{K/\pi} Ratio}}: Higher energy neutrinos (\quickmath{E_{\nu} < 10\text{GeV}}) become dependent upon kaon decay as the dominant source of neutrinos. Measurements of the ratio of \quickmath{K/\pi} \cite{Ambrosini1998-er} are used to control the systematic uncertainty of the expected ratio of pion and kaon production.
\item \textbf{Solar Activity}: As the 11-year solar cycle can affect the Earth's magnetic field, the flux of primary cosmic rays is modulated across the same period. The uncertainity is calculated by taking a \quickmath{\pm 1} year variation, equating to a \quickmath{10\%} uncertainty for the SK-IV period.
\item \textbf{Atmospheric Density}: The height of the interaction of the primary cosmic rays is dependent upon the atmospheric density. The HKKM assumes the US standard 1976 \cite{USStandardAtm} profile. This systematic controls the uncertainty in that model.
\end{itemize}

Updates to the HKKM and Bartol models are underway to use a similar tuning technique to that used in the beam flux predictions. After those updates, it may be possible to include correlations in the hadron production uncertanty systematics for beam and atmospheric flux predictions.

\subsection{Neutrino Interaction}
\label{sec:SelsAndSysts_Systs_Interaction}

The neutrino interactions which occur within all the detectors are modelled by NEUT. The two indpendent oscillation analyses, T2K beam only and the SK atmospheric only, have develop separate interaction models. The T2K-only analysis uses the systematics model defined in \cite{t2k_tn_344} and the SK-only analysis uses the uncertainties detailed in \cite{Kamiokande_Collaboration2017-nf}. To leverage the most sensitivity out of this joint beam and atmospheric analysis, a correlated interaction model has been defined. Where applicable, these correlations allow the systematic uncertainties applied to the atmospheric samples to be constrained by measurements of the near detector in the beam experiment leading to stronger sensitivity to oscillation parameters as compared to an uncorrelated model. An in-depth discussion of the reasoning and validaity of enforcing correlations is documented in \cite{t2k_tn_422} and briefly summarised below.

The low energy T2K systematic model has a more sophisticated treatment of CCQE, CCMEC and CCRES uncertainties which is due to the purpose made cross-section measurements made by the near detector. Furtermore, extensive testing of this model has been performed by the working group responsible for this model \cite{t2k_tn_344}. However, it is not designed for the high energy atmospheric events illustrated in \autoref{fig:Simulations_NeutrinoEnergyDistribution}. Therefore the high energy systematic model from the SK-only analysis is implemented for the relevent multiGeV samples. The CCQE systematic parameters invoked within the SK high energy model are actually contained within T2K's CCQE model. Consequently, the more sophisticated CCQE and CCMEC T2K model parameters have been incorporated into the high energy model but are uncorrelated from the low energy counterparts. This results in a more complete model but without any constraint from the near detector measurements.

The high energy systematic model includes parameters developed from comparisons of Nieves and Rein-Seghal models which affect CCRES interactions, comparisons of the GRV98 and CKMT models which control DIS interactions, and hadron multiplicity measurements which modulate the normalisation of CC\quickmath{N\pi} events. The uncertainty of the \quickmath{\nu_{\tau}} cross-section is particularly large and is controlled by a \quickmath{25\%} normalisation uncertainty. These parameters are applied via normalisation or shape parameters. The former linearly scales the weight of all effected Monte-Carlo events, whereas the latter can increase or decrease a particular events weight depending on its neutrino energy and mode of interaction. The response of the shape parameters are defined by third order polynominal splines which return a weight for a particular neutrino energy. In total, \quickmath{17} normalisation and \quickmath{15} shape parameters are included in the more sophisticated high energy model.

\autoref{fig:SelsAndSysts_NeutrinoEnergyComparison} indicates the predicted neutrino energy distibution for both beam and subGeV atmospheric samples, and \autoref{fig:SelsAndSysts_FractionalModeComparison} illustrates the fractional contribution of the different interaction modes per sample. There is clearly significant overlap in neutrino energy between the subGeV atmospheric and beam samples, allowing similar kinematics in the final state particles. Comparing beam samples with zero decay electrons and atmospheric electron-like(muon-like) samples with zero(zero or one) decay electrons, there is a very similar contribution of CCQE, CC 2p2h and CC\quickmath{1\pi^{\pm}} interactions. The samples which target CC\quickmath{1\pi^{\pm}} interactions, FHC 1R\quickmath{e}1de beam sample and atmospheric electron-like(muon-like) samples with one(two) decay electrons, also consist of very similar mode interactions. As a consequence of the similarity in energy and mode contributions, correlating the systematic model between the beam and subGeV atmospheric samples ensures that this analysis attains the largest sensitivity to oscillation parameters while still ensuring neutrinio interaction systematics are correctly accounted for. Due to its sophisticated CCQE model, the T2K systematic model was chosen as the basis of the correlated model. 

\begin{figure}[h]
  \begin{subfigure}[t]{0.49\textwidth}
    \includegraphics[width=\textwidth, trim={0mm 0mm 0mm 0mm}, clip,page=1]{Figures/Selections/NeutrinoEnergyDist_Comp_1Rmu_NuMu.pdf}
    \subcaption{\quickmath{\mu}-like}
  \end{subfigure}%
  \begin{subfigure}[t]{0.49\textwidth}
    \includegraphics[width=\textwidth, trim={0mm 0mm 0mm 0mm}, clip,page=1]{Figures/Selections/NeutrinoEnergyDist_Comp_1Re_NuE.pdf}
    \subcaption{\quickmath{e}-like}
  \end{subfigure}
  \begin{subfigure}[t]{0.49\textwidth}
    \includegraphics[width=\textwidth, trim={0mm 0mm 0mm 0mm}, clip,page=1]{Figures/Selections/NeutrinoEnergyDist_Comp_1Re1de_NuE.pdf}
    \subcaption{\quickmath{e}-like + 1d.e.}
  \end{subfigure}
  \caption{The prediction neutrino energy distribution for subGeV atmospheric and beam samples, given for muon-like samples FHC+RHC 1R\quickmath{\mu} beam samples compared to the subGeV \quickmath{\mu}-like 0+1 decay electrons (d.e.) atmospheric samples, electron-like 0d.e. samples FHC+RHC 1R\quickmath{e} beam samples compared to the subGeV \quickmath{e}-like 1d.e. sample, and electron-like 1d.e. sample FHC1Re1de beam sample compared to the subGeV \quickmath{e}-like 1d.e. atmospheric sample.}
  \label{fig:SelsAndSysts_NeutrinoEnergyComparison}
\end{figure}

\begin{figure}[h]
  \begin{subfigure}[t]{\textwidth}
    \includegraphics[width=\textwidth, trim={0mm 0mm 0mm 0mm}, clip,page=1]{Figures/Selections/FractionalModeComparison.pdf}
  \end{subfigure}
  \caption{The interaction mode contribution of each sample given as a fraction of the total event rate in that sample. All systematic dials are set to their nominal values and the Asimov A oscillation parameters are assumed. The Charged Current (CC) modes are broken into quasi-elastic (QE), meson exchange (2p2h), resonant charged pion production (\quickmath{1\pi^{\pm}}), multi-pion production (\quickmath{M\pi}), and other interaction category. Neutal Current (NC) interaction modes are given in interaction mode categories: \quickmath{\pi^{0}} production, resonant charged pion production,  multi-pion production and other.}
  \label{fig:SelsAndSysts_FractionalModeComparison}
\end{figure}

The T2K uncertainty model is applied in a similar methodology to the SK model parameters. It consists of \quickmath{19} shape parameters applied via third order polynominal splines and \quickmath{24} normalisation parameters. Four additional parameters, which model the uncertainty in the bining energy, are applied in a way to shift the momentum of lepton emitted from a nucleus. The majority of these parameters are assigned a Gaussian prior uncertainty. Those that have no theoretical reasoning, or those which have not been fit to external data, are assigned a flat prior which does not affect the penalty term. The CCQE model parameters were tuned to MiniBooNE \cite{miniboone_nu_ccqe} and MINER\quickmath{\nu}A \cite{minerva_nubar_ccqe} measurements and CCRES model parameters are tuned to ANL and BNL experiments \cite{ANL_BNL_corr}.

There are three particular tunes of the T2K low energy cross section model typically considered. Firstly, the ``generated'' tune which is the set of dial values at which the Monte Carlo was generated with. Secondly, the set of dial values which are taken from external data measurements and used as inputs. These are the ``pre-fit'' dial values. The reason these two sets of dial values are different is because the external data measurements are continually updated but it is very computational intensive to regenerate a Monte Carlo prediction after each update. Consequently the pre-fit and generated dial values differ. The final tune is the ``post-fit'' or ``post-BANFF'' dial values. These are the values taken from a fit to the beam near detector data. This fit is performed by two indepedent fitting frameworks, \texttt{MaCh3} and \texttt{BANFF}, which ensures reliable measurements. The output of each fitter is converted into a covariance matrix to describe the error and correlations between all the cross section parameters. This is then propagated to the far-detector oscillation analysis group for use in the \texttt{P-Theta} fitting framework. As \text{MaCh3} can perform a near detector fit, it is included within the simulataineous fit of far-detector beam and atmospheric oscillation analysis. This is because this technique does not require any assumption of Gaussian posterior distributions which is required in the covariance matrix methodology.

On top of the combination of the SK and T2K interaction models, several other parameters have been specifically developed for the joint oscillation analysis. As the majority of the atmospheric samples' \dcp sensitivity comes from the normalisation of subGeV electron-like events, additional dial which models an alternative Continous Random Phase Approximation (CRPA) nuclear ground state has been implemented \cite{t2k_tn_422}. As the near detector can not sufficiently constrain the model, this dial approximates the event weights if a CRPA model had been assumed rather than a spectral function. This dial only effects \quickmath{\nu_{e}} and \quickmath{\bar{\nu}_{e}} and is applied as a shape parameter.

Further additions to the model have been included due to the the subGeV \quickmath{\pi^{0}} atmospheric sample. This particulary targets charged current and neutral current \quickmath{\pi^{0}} producing interactions to constrain the systematic uncertainties. However, there is no analogous sample in the T2K beam-only analysis so no significant effort has been placed into building a sufficient uncertainty model. Therefore, an uncertainty which effects neutral current resonant \quickmath{\pi^{0}} production is incorporated in this analysis. Comparisons of NEUT's NC resonant pion production predictions have been made to MiniBooNE \cite{MB_NC1pi0} data and a consistent \quickmath{16\%} to \quickmath{21\%} underprediction is observed. Consequently, a conservative \quickmath{30\%} normalisation is invoked. 

Events which originate from above the detector and travel downward are very insensitive to oscillation parameters and act similar to the near detector within an accelerator experiment (Details are illustrated in \autoref{chap:OscillationProbability}). Consequently, the application of the T2K low energy cross section and the effect of the near detector constraint on the atmospheric samples can be studied through these events, without biasing the results from oscillation effects. The downward going predictives are illustrated in \autoref{fig:SelsAndSysts_DownGoingPredictives}. For samples which target CCQE interactions (electron-like with 0 decay electrons and muon-like with 0 or 1 decay electrons), the application of the near detector constraint is well within statistical fluctuation of the down-going data and no significant tension is observed between the data and the Monte Carlo prediction with the T2K near detector constraint. This is not the case for samples with target CCRES interactions (electron-like with 1 decay electron and muon-like with 2 deca electrons). The electron-like data is cosistent with the constrained prediction at higher reconstructed momenta but diverges at lower momentum, whereas the muon-like sample is under-predicted throughout the range of momenta. To combat this disagreement, an additional cross section systematic dial, specifically designed to inflate the low pion momentum systematics was developed in \cite{t2k_tn_422}. This is a shape parameter implemented with through a splined response.

\begin{figure}[h]
  \begin{subfigure}[t]{0.49\textwidth}
    \includegraphics[width=\textwidth, trim={0mm 0mm 0mm 0mm}, clip,page=1]{Figures/Selections/Predictive_SubGeV-elike-0dcy.pdf}
  \end{subfigure}%
  \begin{subfigure}[t]{0.49\textwidth}
    \includegraphics[width=\textwidth, trim={0mm 0mm 0mm 0mm}, clip,page=1]{Figures/Selections/Predictive_SubGeV-elike-1dcy.pdf}
  \end{subfigure}
  \begin{subfigure}[t]{0.49\textwidth}
    \includegraphics[width=\textwidth, trim={0mm 0mm 0mm 0mm}, clip,page=1]{Figures/Selections/Predictive_SubGeV-mulike-0dcy.pdf}
  \end{subfigure}%
  \begin{subfigure}[t]{0.49\textwidth}
    \includegraphics[width=\textwidth, trim={0mm 0mm 0mm 0mm}, clip,page=1]{Figures/Selections/Predictive_SubGeV-mulike-1dcy.pdf}
  \end{subfigure}
  \begin{subfigure}[t]{0.49\textwidth}
    \includegraphics[width=\textwidth, trim={0mm 0mm 0mm 0mm}, clip,page=1]{Figures/Selections/Predictive_SubGeV-mulike-2dcy.pdf}
  \end{subfigure}
  \caption{Down-going atmospheric subGeV single-ring samples comparing the mean and error of the pre-fit and post-fit Monte Carlo predictions in red and blue, respectively. The magneta histogram illustrates the Monte Carlo prediction using the generated dial values. The black points illustrates the down-going data with statistical errors given. The mean and errors of the Monte Carlo predictions are calculated by the techniques documented in \autoref{sec:MarkovChainMonteCarlo_Predictives}. The cross-section and atmospheric flux parameters are either thrown from their pre-fit uncertainties (denoted pre-fit band), or the cross section dial values are randomly sampled from a MCMC chain whilst the atmospheric flux parameters are thrown from their pre-fit distributions (denoted post-fit band).}
  \label{fig:SelsAndSysts_DownGoingPredictives}
\end{figure}

\subsection{Near Detector}
\label{sec:SelsAndSysts_Systs_ND}

The systematics applied due to uncertainties arising from the response of the near detector is contained within \quickmath{574} normalisation parameters binned in momentum and angle, \quickmath{P_{\mu}} ad \quickmath{\cos(\theta_{\mu})}, of the final-state muon. These are applied via a covariance matrix with each parameter been assigned a Gaussian prior from that covariance matrix. These normalisation parameters are built from underlying systematics, e.g. pion secondary interaction systematics, which are randomly thrown and the variation in each \quickmath{P_{\mu} \times \cos(\theta_{\mu})} bin is determined. This is performed \quickmath{2000} times and a covariance matrix response is created. This allows singificant correlations between FGD1 and FGD2 samples, as well as adjacent bins. Statistical uncertainties are accounted for by including fluctuations of each event's weight from a Poission distribution.

Similar to the cross section systematcs, MaCh3 and BANFF are used to constrain the uncertainty of these systematics through independent validations. Each fitter generates a post-fit covariance matrix which is compared and passed to the far-detector oscillation analysis working group. However, as the analysis presented within this thesis uses the MaCh3 framework, a joint oscillation analysis fit of all three sets of samples is performed. From the T2K-only perspective, this joint analysis including atmospheric samples allows additional constraints on the systematic uncertainties where correlations have been invoked. 

\subsection{Far Detector}
\label{sec:SelsAndSysts_Systs_FD}

Two configurations of the far detector systematic model implementation have been considered. Firstly, the far detector systematic uncertatinties for beam and atmospheric samples are taken from their respective analysis inputs, denoted ``official inputs'' analysis. Consequently, no correlations are assumed between the beam and atmospheric samples. The generation of the beam- and atmospheric- specific inputs are documented in \autoref{sec:SelsAndSysts_Systs_FDBeam} and \autoref{sec:SelsAndSysts_Systs_AtmBeam} for the beam and atmospheric samples, respectively. Secondly, a correlated detector model has been considered. Here, the distribution of parameters used for applying event cuts (e.g. electron-muon separation) are modified within the fit, following similar methodology to the beam far detector systematics model implementation. However, it has been designed to ensure that the atmospheric data is not double-counted, which would be the case for the official inputs analysis. This alternative implementation is detailed in \autoref{sec:SelsAndSysts_Systs_Correlated}.

\subsubsection{Beam Samples}
\label{sec:SelsAndSysts_Systs_FDBeam}

There are \quickmath{45} systematics which describe the response of the far detector, specifically for beam sample neutrino events. \quickmath{44} of these parameters are normalisation parameters and are split by the interaction mode, true neutrino flavour, reconstructed neutrino energy and sample which they effect. The final parameter is the energy scale uncertainty. It is applied as a multiplicative factor to the reconstructed neutrino energy. The value of the systematic is taken from Monte Carlo to data differences illustrated in \cite{sk_2017}. The normalisation parameters are assigned a Gaussian error centrelised at \quickmath{1.0} and error taken from a covariance matrix. A detailed breakdown of the following procedure is found in \cite{t2k_tn_318}. To build the covariance matrix, first a fit is performed to atmospheric data which has been selected using beam sample selection cuts. The variable which defines each cut, \quickmath{L} (e.g. the electron-muon pid parameter) is assigned a smear, \quickmath{\alpha}, and shift, \quickmath{\beta} parameter such that,

\begin{equation}
  \label{eqn:SelsAndSysts_Systs_ShiftSmear}
  L^{i}_{j} \rightarrow \bar{L}^{i}_{j} = \alpha^{i}_{j} L + \beta^{i}_{j}
\end{equation}

Where \quickmath{L^{i}_{j}} (\quickmath{\bar{L}^{i}_{j}}) correspond to nominal(varied) pid cut parameters given in \autoref{tab:SelsAndSysts_Systs_CutVariables}. The shift and smear parameters are binned by final-state topology, \quickmath{j}, where the binning is given in \autoref{tab:SelsAndSysts_Systs_Topologies}. This approach is uses to allow the cut parameter distributions to be modified within the fit which allows better data to Monte Carlo agreement.

\begin{table}[ht!]
    \centering
    \begin{tabular}{c|l}
      \hline
      Cut Variable & Parameter Name \\
      \hline
      \quickmath{0} & \fq \quickmath{e/\mu} PID \\
      \quickmath{1} & \fq \quickmath{e/\pi^{0}} PID \\
      \quickmath{2} & \fq \quickmath{\mu/\pi} PID \\
      \quickmath{3} & \fq Ring-Counting Parameter \\
      \hline
      \hline
    \end{tabular}
    \caption{List of cut variables which are included within the shift/smear fit documented in \cite{t2k_tn_318}.}      
    \label{tab:SelsAndSysts_Systs_CutVariables}
\end{table}

\begin{table}[ht!]
  \resizebox{16cm}{!}{
    \centering
    \begin{tabular}{c|c|l}
      \hline
      Index & Category & Description \\
      \hline
      \quickmath{0} & \quickmath{1e} & Only one electron above Cerenkov threshold in the final state \\
      \quickmath{1} & \quickmath{1\mu} & Only one muon above Cerenkov threshold in the final state \\
      \quickmath{2} & \quickmath{1e\text{+other}} & One electron and one or more other charged particles above Cerenkov threshold in the final state \\
      \quickmath{3} & \quickmath{1\mu\text{+other}} & One muon and one or more other charged particles above Cerenkov threshold in the final state \\
      \quickmath{4} & \quickmath{1\pi^0} & Only one $\pi^0$ in the final state\\
      \quickmath{5} & \quickmath{1\pi^\pm} or \quickmath{1\text{p}} & Only one hadron (typically charged pion or proton) in the final state\\
      \quickmath{6} & Other & Any other final state\\
      \hline
    \end{tabular}
  }
  \caption{Reconstructed event topology categories on which the SK detector systematics \cite{t2k_tn_318} are based.}
  \label{tab:SelsAndSysts_Systs_Topologies}
\end{table}

Beyond the uncertainty on the pid cut criteria, the mis-modelling of \quickmath{\pi^{0}} events is also considered. If one of the two rings from a \quickmath{\pi^{0}} event is missed, this will reconstructed as a CC\quickmath{\nu_{e}} event. This is one of the largest systematics hindering the electron neutrino appearance analyses. Consequently, a systematic has been introduced to constrain the mis-modelling of \quickmath{\pi^{0}} events in SK. To evaluate this systematic uncertainty, a set of “hybrid-\quickmath{\pi^{0}} samples is constructed. These events are built by overlaying one electron-like ring from the SK atmospheric neutrino samples or decay electron ring from a stopping cosmic ray muon with one simulated photon ring. Both rings are chosen so that momenta and opening angle follow the decay kinematics of NC \quickmath{\pi^0} events from the T2K-MC. Hybrid-\quickmath{\pi^{0}} Monte Carlo samples with both rings from the SK Monte Carlo are produced to compare with the hybrid-\quickmath{\pi^{0}} data samples and the difference in the fraction of events that pass the \quickmath{\nu_{e}} selection criteria is used to assign the systematic error. In order to investigate any data to Monte Carlo differences which may originate from either the higher energy ring or lower energy ring, two samples are built; a sample in which the electron constitutes the higher energy ring from the \quickmath{\pi^{0}} decay called the primary sample, and another one in which it constitutes the lower energy ring called the secondary sample. The standard T2K \quickmath{\nu_{e}} fiTQun event selection criteria are used to select events.

Final contributions to the covariance matrix are determined by supplementary uncertainties attained by comparing stopping muon data to Monte Carlo prediction, as first introduced in \autoref{sec:Simulation_Reconstruction}. The efficiency of tagging decay electrons is estimated by the stopping muon data/Monte Carlo differences by comparing the number of one decay electron events to the number of events with one or less decay electrons. The rate at which fake decay electrons are reconstructed by \fq is estimated in a similar way with the only difference being the ratio compares the number of two decay electron events to the number of events with one or two reconstructed decay electrons. The two sources of systematics are added in quadrature weighted by the number of events with one true decay electron yielding a \quickmath{0.2\%} systematic uncertainty. The muon mis-identification rate is estimated by comparing the number of electron-like events which have one decay electron to the total number of events with one decay electron. This systematic is estimated as a \quickmath{30\%} effect in the rate of muon mis-identification. A fiducial volume systematic of \quickmath{\pm 2.5\text{cm}} which corresponds to a \quickmath{0.5\%} shift in the normalisation of events. Additional normalisation uncertainties based on neutrino flavour and interaction mode are also defined in \cite{t2k_tn_326, t2k_tn_186, t2k_tn_107}.

This covariance matrix is then added in quadrature with two other covariance matrices. These are matrices which describe the uncertainties due to secondary interactions which modify the final state kinematics and the photo-nucleon interactions. These are generated by studying the effect of each samples event rates when considering variations of the underlying parameters.

\subsubsection{Atmospheric Samples}
\label{sec:SelsAndSysts_Systs_AtmBeam}

The systematic parameters which control the detector systematics are split into two sub-groups. Those which are related to particle identification and ring counting systematics and those which are related to calibration measurements. 

The particle identification systematics consist of five parameters. The ring separation systematic enforces an anti-correlated response between the single-ring and multi-ring samples. This is implemented as a fractional increase/decrease in the overall normalisation of each sample, depending on the distance to the nearest wall from an event's vertex. The coefficients of the normalisation is estimated prior to the fit and depends on the atmospheric sample. The single-ring and multi-ring PID systematics encode the detector's ability to separate electron-like and muon-like events and are implemented in an identical way as the ring separation systematic.

The multi-ring electron-like separation systematics encode the ability of the detector to separate neutrino from anti-neutrino events. As an important systematic in the mass hierarchy determination, this systematic controls the relative normalisation's of the \quickmath{\nu_e} and \quickmath{\bar{\nu}_e} enriched samples. A two-stage approach is implemented in the event selection and a systematic is implemented for both stages. The first stage in the event selection is to confirm that the most energetic ring, which is required to be electron-like, is from the neutrino interaction rather than a pion decay from any hadronic system present in the event. The second stage of separation uses a likelihood-based cut to separate \quickmath{\nu_e/\bar{\nu}_e} events. This takes the typical properties of \quickmath{\nu_e} scattering events into account; e.g. less forward-going, with larger energy fractions in the hadronic system. These parameters are implemented via normalisation parameters which vary the event rate of each multi-ring sample, whilst ensuring the total event rate is conserved.

There are 22 systematics related to calibration measurements, including effects from backgrounds, reduction and showering effects. They are documented in \cite{Jiang2019-iw} are briefly summarised in \autoref{SelsAndSysts_Systs_ATMPDCalibSysts}. They are applied via normalisation parameters, with the separation systematics required the conservation of event rate across all samples.

\begin{table}[ht!]
  \centering
  \caption{Sources of systematic errors specified within the grouped into the ``calibration'' systematics model.}
  \label{SelsAndSysts_Systs_ATMPDCalibSysts}
		
  \begin{tabular}{ll} 
    
    \hline Index & Description \\ 
    \hline
    0 & Partially contained reduction  \\
    1 & Fully contained reduction \\
    2 & Separation of fully contained and partially contained events \\
    3 & Separation of stopping and through-going\\
      & \hspace{0.4cm} partially contained events in top of detector \\
    4 & Separation of stopping and through-going\\
      & \hspace{0.4cm} partially contained events in barrel of detector \\
    5 & Separation of stopping and through-going\\
      & \hspace{0.4cm} partially contained events in bottom of detector \\
    6 & Background due to cosmic rays \\
    7 & Background due to flasher events \\
    8 & Vertex systematic moving events into and out of fiducial volume \\
    9 & Upward going muon event reduction \\
    10 & Separation of stopping and through-going in upward going muon events \\
    11 & Energy systematic in upward going muon events \\
    12 & Reconstruction of path length of upward going muon events \\
    13 & Separation of showering and non-showering upward going muon events \\
    14 & Background of stopping upward going muon events \\
    15 & Background of non-showering through-going upward going muon events \\
    16 & Background of showering through-going upward going muon events \\
    17 & Efficiency of tagging two rings from $\pi^0$ decay \\
    18 & Efficiency of decay electron tagging \\
    19 & Background from down going cosmic muons \\
    20 & Asymmetry of energy deposition in tank \\
    21 & Energy scale deposition \\
    \hline
  \end{tabular}
\end{table}

\subsubsection{Correlated Detector Model}
\label{sec:SelsAndSysts_Systs_Correlated}

A complete uncertainty model of the SK detector would be able to determine the systematic shift on the sample spectra for a variation of the underyling parameters, e.g. PMT angular acceptance. However, this is particularly resource intensive, requiring Monte Carlo predictions to be made for each plausible variation. Consequently an effective parameter model has been utilised for a correlated detector model. This follows from the T2K-only model implementation documented in \autoref{sec:SelsAndSysts_Systs_FDBeam}. The T2K-only implementation can not be used for atmospheric sample systematics because it is built upon on a fit to atmospheric data. Consequently, an implementation where the cut distributions (given in \autoref{tab:SelsAndSysts_Systs_CutVariables}) from both beam and atmospheric samples are fit, whilst simultaneously fitting for oscillation parameters. The fit to the cut variables performs a shape-only fit to ensure that no double-counting occurs.

The correlated detector model utilises the same smear and shift parameters documented in \autoref{sec:SelsAndSysts_Systs_FDBeam}, split by final state topology. This splitting is done because the detector will respond differently for events which have one or multiple rings. For example, the detector will be able to distinguish single-ring events better than two overlapping ring events, resulting in smaller systematic uncertainty for one ring events compared to two ring events. Furthermore, the shift and smear parameters are split by visible energy deposited within the tank has been included, with binning specified in \autoref{tab:SelsAndSysts_Systs_EVisBinning}. This is because atmospheric events are categorised by subGeV and multiGeV events based on visible energy, so this splitting is required when correlating the systematic model for beam and atmospheric events. Alongside the technical requirement, higher energy events will be better recontructed due to fractionally less noise within the detector. Consequently, this analysis correlates the detector systematics between the far-detector beam and subGeV atmospheric samples due to their similar energies and interaction types. As a result of the incluson of visible energy binning, \autoref{eqn:SelsAndSysts_Systs_ShiftSmear} becomes

\begin{equation}
  \label{eqn:SelsAndSysts_Systs_ShiftSmearWithEVis}
  L^{i}_{jk} \rightarrow \bar{L}^{i}_{jk} = \alpha^{i}_{jk} L + \beta^{i}_{jk},
\end{equation}

where \quickmath{k} is the visible energy bin. The multi-GeV, multi-ring, PC and Up-\quickmath{\mu} samples will be subject to the ATMPD particle identification systematics implementation as described in \autoref{sec:SelsAndSysts_Systs_AtmBeam} rather than using this correlated detector model. The calibration systematics also described in the aforementioned chapter still apply to all atmospheric samples.

\begin{table}[ht!]
    \centering
    \begin{tabular}{c|c}
      \hline
      Index & Range (MeV) \\
      \hline
      \quickmath{0} & \quickmath{30 \geq x > 300} \\
      \quickmath{1} & \quickmath{300 \geq x > 700} \\
      \quickmath{2} & \quickmath{700 \geq x > 1330} \\
      \quickmath{3} & \quickmath{1330 \geq x} \\
      \hline
      \hline
    \end{tabular}
    \caption{Reconstructed event topology categories on which the SK detector systematics are based}
    \label{tab:SelsAndSysts_Systs_EVisBinning}
\end{table}

The implementation of this systematic model takes the events reconstructed values of the cut parameters, modifies them by the particular shift and smear parameter for that event, and then re-applies event selection. This invokes event migration, which is a new feature incorporated into the MaCh3 framework which is only achievable due to the event-by-event reweighting scheme.

Particular care has to be taken when varying the ring counting parameter. This is because the number of rings is a finite value (one-ring, two-rings, etc.) which can not be continously varied. Consequently a ring counting parameter, \quickmath{RC_{i}}, is calculated for the \quickmath{i^{th}} event, following the definition in \cite{Tobayama:2016dsi}. The likelihood from all considered one-ring (\quickmath{1R}) and two-ring (\quickmath{2R}) fits are compared to determine the preferred hypothesis. This is done by searching for the minimum log likelihoods, \quickmath{\log(L_{1R})} and \quickmath{\log(L_{2R})}. The difference is computed as \quickmath{\Delta_{LLH} = \log(L_{1R}) - \log(L_{2R})}. The ring counting parameter is then defined as,

\begin{equation}
  \label{eqn:SelsAndSysts_Systs_RCParam}
  RC_{i} = \text{sgn} \left(\Delta_{LLH} - C_{Thres} \right) \times \sqrt{\lvert \Delta_{LLH} - C_{Thres} \rvert},
\end{equation}

where \quickmath{C_{Thres} = 150.0 - 0.6 \times P_{2R}}, and \quickmath{P_{2R}} is the momentum of the preferred two-ring hypothesis, and \quickmath{\text{sgn}(x) = x/\lvert x \rvert}. The co-efficients used within the definition of \quickmath{C_{Thres}} are calculated based Monte Carlo studies. This ring counting parameter corresponds to a intermediate likelihood value used within the \fq algorithm to decide the number of rings associated with a particular event. However, fake-ring merging algorithms are applied after this likelihood value is used to determine the number of rings associated with an event. Consequently, this ring counting parameter does not always exactly correspond to the number of reconstructed rings. This can be seen in \autoref{fig:SelsAndSysts_RCParameterDistribution}.

\begin{figure}[h]
  \begin{subfigure}[t]{0.49\textwidth}
    \includegraphics[width=\textwidth, trim={0mm 0mm 0mm 0mm}, clip,page=1]{Figures/Selections/RCParam_SGE0Dcy.pdf}
    \subcaption{SubGeV e-Like 0d.e.}
  \end{subfigure}%
  \begin{subfigure}[t]{0.49\textwidth}
    \includegraphics[width=\textwidth, trim={0mm 0mm 0mm 0mm}, clip,page=1]{Figures/Selections/RCParam_MRENue.pdf}
    \subcaption{MultiRing \quickmath{\nu_{e}}-Like}
  \end{subfigure}
  \begin{subfigure}[t]{0.49\textwidth}
    \includegraphics[width=\textwidth, trim={0mm 0mm 0mm 0mm}, clip,page=1]{Figures/Selections/RCParameterLegend.pdf}
  \end{subfigure}
  \caption{The ring counting parameter, as defined in \autoref{eqn:SelsAndSysts_Systs_RCParam}, for the subGeV electron-like zero decay electron and multi-ring \quickmath{\nu_{e}}-like samples.}
  \label{fig:SelsAndSysts_RCParameterDistribution}
\end{figure}

As the \fq algorithm does not provide a likelihood value after the fake-ring algorithms have been applied, the ring counting parameter distribution is connected to the final number of reconstructed rings through ``maps''. These are two dimensional distributions linking the ring counting parameter and the final number of reconstructed rings. An example is illustrated in \autoref{fig:SelsAndSysts_RCMaps}. In principle, the \fq reconstruction algorithm should be re-ran after the variation in the ring counting parameter. However, this is not computationally viable. Therefore the ``maps'' are used as a reweighting template.

The maps are split by final state topology and true neutrino flavour and all \fq-reconstructed Monte Carlo events are used to fill them. To ensure conservation of event rate, the maps are normalised such that the total event rate across all number of reconstructed rings is equal to one. Prior to the fit, an event's nominal weight is calculated as \quickmath{W(N^{i}_{Rings},L^{i}_{jk})}, where \quickmath{N^{i}_{Rings}} is the reconstructed number of rings for the \quickmath{i^{th}} event and \quickmath{W(x,y)} is the bin content in the associated map for \quickmath{x} number of rings and ring counting parameter \quickmath{L}. Then during the fit, the value of \quickmath{R = W(N^{i}_{Rings},\bar{L}^{i}_{jk})/W(N^{i}_{Rings},L^{i}_{jk})} is calculated as the ring-counting weight for the \quickmath{i^{th}} event. This is the only cut variable which uses a reweighting scheme rather than event migration.

\begin{figure}[h]
  \begin{subfigure}[t]{0.49\textwidth}
    \includegraphics[width=\textwidth, trim={0mm 0mm 0mm 0mm}, clip,page=1]{Figures/Selections/NuFlavour_14_Top_1.pdf}
  \end{subfigure}%
  \begin{subfigure}[t]{0.49\textwidth}
    \includegraphics[width=\textwidth, trim={0mm 0mm 0mm 0mm}, clip,page=1]{Figures/Selections/NuFlavour_14_Top_3.pdf}
  \end{subfigure}
  \caption{The ring counting parameter, defined in \autoref{eqn:SelsAndSysts_Systs_RCParam}, as a function of the number of reconstructed rings as found by the \fq algorithm. Left: true \quickmath{\nu_{\mu}} events with only one muon above Cherenkov threshold in the final state. Right: true \quickmath{\nu_{\mu}} events with one mmuon and at least one other charged particle above Cherenkov threshold in the final state.}
  \label{fig:SelsAndSysts_RCMaps}
\end{figure}

The \quickmath{\pi^{0}} systematics introduced in \autoref{sec:SelsAndSysts_Systs_ND} were expected to be applied via a covariance matrix. As this alternative technique performs a simultaneous fit between detector distributions and oscillation parameters, the implementation of the \quickmath{\pi^{0}} systematics has been modified. In practice, the inputs from the hybrid \quickmath{\pi^{0}} sample is included via the use of ``\quickmath{\chi^{2}} maps'', which are two dimensional histograms in \quickmath{\alpha} and \quickmath{\beta} parameters over some range. Illustrative examples of the \quickmath{\chi^{2}} maps are given in \autoref{fig:SelsAndSysts_HybridChi2Maps}. Due to their nature, the shift and smear parameter are typically very correlated.

The maps are filled through the \quickmath{\chi^{2}} comparison of the hybrid \quickmath{\pi^{0}} Monte Carlo and data in the particle identification parameters documented in \autoref{tab:SelsAndSysts_Systs_CutVariables}. The Monte Carlo distribution is modified with the \quickmath{\alpha} and \quickmath{\beta} scaling, whilst cross-section and flux nuisance parameters are thrown from there prior uncertainties, and the \quickmath{\chi^{2}} between the scaled Monte Carlo and data is calculated and the relevant point in the \quickmath{\chi^{2}} map is filled. Then in the fit, the likelihood penalty term is found for the particular particle identification parameter by using the value of the relevant \quickmath{\chi^{2}} map for the \quickmath{\alpha} and \quickmath{\beta} parameter at that step in the MCMC fit. For this fit, only \quickmath{1\pi^{0}} final state topology shift and smear parameters use the hybrid \quickmath{\pi^{0}} \quickmath{\chi^{2}} prior uncertainty. 

\begin{figure}[h]
  \begin{subfigure}[t]{0.49\textwidth}
    \includegraphics[width=\textwidth, trim={0mm 0mm 0mm 0mm}, clip,page=1]{Figures/Selections/EMUPID_Elt300_Chi2Map.pdf}
  \end{subfigure}%
  \begin{subfigure}[t]{0.49\textwidth}
    \includegraphics[width=\textwidth, trim={0mm 0mm 0mm 0mm}, clip,page=1]{Figures/Selections/EPI0PID_Elt300_Chi2Map.pdf}
  \end{subfigure}
  \caption{The \quickmath{\chi^{2}} between the hybrid-\quickmath{\pi^{0}} Monte Carlo and data samples, as a function of smear (\quickmath{\alpha}) and shift (\quickmath{\beta}) parameters, for events which have \quickmath{1\pi^{0}} final state topology. Left: Electron-muon separation PID parameter for events with \quickmath{30 \geq E_{vis} (MeV) < 300}. Right: Electron-\quickmath{\pi^{0}} separation PID parametter for events with \quickmath{30 \geq E_{vis} (MeV) < 300}.}
  \label{fig:SelsAndSysts_HybridChi2Maps}
\end{figure}

Similarly, the supplementary systematics which are added into the covariance from stopping muon and decay electron studies need to be included. A new framework \cite{t2ksk-common} was built in tandem with the T2K-SK working group \cite{t2k_tn_326} so the additional parameters can be incorporated in the MaCh3 framework. These are applied as normalisation parameters, depending on the partricular interaction mode, number of tagged decay electrons and whether the primary particle generated Cherenkov light. They are assigned Gaussian uncertainties with widths described by a covariance matrix.

Finally, the secondary interaction and photo-nuclear effects need to be accounted for in this detector model. In the T2K-only analysis, a covariance matrix was built to describe the response of the samples to variations of these parameters which was then added in quadrature to the detector covariance matrix. However, this technique can not be applied in the correlated detector model. Consequently, a binned response of each of the secondary interaction systematic parameters and the photo-nuclear response was generated and included through splined shape parameters, similar to the application of shape parameters in the cross-section model (see \autoref{sec:SelsAndSysts_Systs_Interaction}).

There are a total of \quickmath{224} \quickmath{\alpha^{i}_{jk}} and \quickmath{\beta^{i}_{jk}} parameters, of which \quickmath{32} have prior constraints from the hybrid \quickmath{\pi^0} samples.

One final complexity of this correlated detector model is that the two sets of samples, beam and subGeV atmospheric, use slightly different parameters to distinguish electron and muon like events. The beam-only events use the \quickmath{\log(L_{e}/L_{\mu})} whereas the atmospheric samples use \quickmath{\log(L_{e}/L_{\pi})}, where \quickmath{L_{X}} is the likelihood for hypothesis \quickmath{X}. This is because the beam-only fits use single-ring \fq fitting techniques, whereas multi-ring fits are applied to the atmospheric samples where only the electron and pion hypothesis are considered. As discussed in \autoref{sec:Simulation_Reconstruction}, the pion hypothesis is a very good approximation of the muon hypothesis due to their similar mass. The correlation between the two likelihood ratios is illustrated in \autoref{fig:SelsAndSysts_LLHCorrelation}. A very strong correlation is clearly shown. Consequently, using the same shift and smear parameters correlated between beam and subGeV atmospheric is a good approximation.

\begin{figure}[h]
  \begin{subfigure}[t]{0.49\textwidth}
    \includegraphics[width=\textwidth, trim={0mm 0mm 0mm 0mm}, clip,page=1]{Figures/Selections/Correlation_SG0Dcy.pdf}
  \end{subfigure}%
  \begin{subfigure}[t]{0.49\textwidth}
    \includegraphics[width=\textwidth, trim={0mm 0mm 0mm 0mm}, clip,page=1]{Figures/Selections/Correlation_SG1Dcy.pdf}
  \end{subfigure}
  \begin{subfigure}[t]{0.49\textwidth}
    \includegraphics[width=\textwidth, trim={0mm 0mm 0mm 0mm}, clip,page=1]{Figures/Selections/Correlation_SG2Dcy.pdf}
  \end{subfigure}
  \caption{The distribution of \quickmath{\log(L_{e}/L_{\mu})} compared to \quickmath{\log(L_{e}/L_{\pi})} for subGeV events with zero (top left), one (top right) or two (bottom) decay electrons. The correlation in the distribution is calculated as \quickmath{0.997}, \quickmath{0.999} and \quickmath{0.996}, respectively.}
  \label{fig:SelsAndSysts_LLHCorrelation}
\end{figure}

