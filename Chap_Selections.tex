\chapter{Simulation, Reconstruction and Event Selections}
\label{chap:Selections}

\section{Simulation}
\label{sec:Selections_Simulation}

In order to generate a Monte Carlo prediction of the expected event rate at the far detector for both sets of samples, all the processes in the beamline, atmospheric flux, neutrino interaction and detector need to be modelled. The beamline simulation consists of three distinct parts; initial hadron interaction modelling, target station geometry and particle tracking and hadronic re-interactions. These are modelled by FLUKA \cite{fluka2011}, JNUBEAM \cite{geant3, PhysRevD.87.012001} and GCALOR \cite{gcalor}, respectively. FLUKA is not very adaptable but matches external cross-section measurements in the region of interest better than GCALOR (\quickmath{O(10)\text{GeV}}). Thus a small simulation is built to model the interactions in the target and the output is then passed to JNUBEAM and GCALOR for propagation. The hadronic interactions are tuned to data from the NA61/SHINE \cite{Abgrall_2011, Abgrall_2012, NA61_pions_rep} and HARP \cite{harp} experiments. The tuning is done by reweighting the FLUKA and GCALOR predictions to match the external data multiplicity and cross-section measurements, based on final state particle kinematics \cite{t2k_tn_flux}. The predicted flux for neutrino and antineutrino beam modes is illustrated in \autoref{fig:T2KSKExp_T2K_NuFluxPerMode}.

The atmospheric neutrino flux predictions are simulated by the HKKM model \cite{Honda_2007, Honda:2011}, where the primary cosmic ray flux is tuned to AMS \cite{Blau2002} and BESS \cite{Haino2004} external data assuming the US-standard atmosphere `76 \cite{USStandardAtm} density profile and includes geomagnetic field effects. Secondary interactions of pions and muons are handled by DPMJET-III \cite{Roesler2001} for energies above \quickmath{32\text{GeV}} and JAM \cite{Niita2006, Honda:2011} for energies below that value. These hadronic interactions are tuned to external data \cite{Sanuki_2002, Achard_2004} using the same methodology as the tuning of the beamline simulation. The energy and cosine zenith predictions of \quickmath{\nu_{e}, \bar{\nu}_{e}, \nu_{\mu}, \bar{\nu}_{\mu}} flux are given in \autoref{fig:NeutrinoOscillationPhysics_AtmosphericNeutrinoFlux} and \autoref{fig:NeutrinoOscillationPhysics_NuFluxZenithAngleDep}, respectively. The flux is approximately symmetrical and peaked around \quickmath{\cos(\theta_{Z}) = 0.0}. This is because horizontally-going pions and kaons can travel further than their vertically-going counterparts resulting in a larger probability of decay to neutrinos. The symmetry is broken in low-energy neutrinos due to geomagnetic effects, which modify the track of the primary cosmic rays.

The neutrino interactions in all three detectors are simulated with NEUT \cite{Hayato2021, neut}. This simulates quasi-elastic (QE), meson exchange (MEC), single meson production (PROD), coherent pion production (COH) and deep inelastic scattering (DIS) interactions. These interaction categories can be further broken down by whether they were propagated via a \quickmath{W^{\pm}} boson in Charged Current (CC) interactions or via a \quickmath{Z^{0}} boson in Neutral Current (NC) interactions. CC interactions have a charged lepton in the final state, which can be flavour-tagged in reconstruction to determine the flavour of the neutrino. In contrast, NC interactions have a neutrino in the final state so no flavour information can be determined from the observables in the detector. This is the reason why NC events are assumed to not oscillate within the analysis. Both CC and NC interactions are modelled for all the above interaction categories, other than MEC interactions which are only modelled for CC events. The SK detector is only sensitive to charged particles, so all charged current interactions are simulated whilst only neutral current processes which produce charged mesons (NCDIS, NCCOH and NCPROD) are modelled. NC MEC interactions can only produce charged particles through secondary re-interactions which is a low cross-section process.

\begin{figure}[h]
  \begin{subfigure}[t]{0.8\textwidth}
    \includegraphics[width=\textwidth, trim={0mm 0mm 0mm 0mm}, clip,page=1]{Figures/Selections/NEUTCrossSection.pdf}
  \end{subfigure}
  \caption{The NEUT prediction of the \quickmath{\nu_{\mu}}-H2O cross-section overlaid on the T2K \quickmath{\nu_{\mu}} flux. The charged current (black, solid) and neutral current (black, dashed) inclusive, charged current quasi-elastic (blue, solid), charged current 2p2h (blue, dashed), charged current single pion production (pink) and charged current multi--\quickmath{\pi} and DIS (Purple) cross-sections are illustrated. Taken from \cite{Hayato2021}}
  \label{fig:Selection_CrossSection}
\end{figure}

As illustrated in \autoref{fig:Selection_CrossSection}, QE interactions dominate the low-energy cross-section of neutrino interactions. The NEUT implementation adopts the Llewellyn Smith \cite{llewelyn-smith} model for neutrino-nucleus interactions, where the nuclear ground state of any bound nucleons (neutrino-oxygen interactions) is approximated by a spectral-function \cite{Benhar1989} model that simulates the effects of Fermi momentum and Pauli blocking. The cross-section of QE interactions are controlled by vector and axial-vector form factors parameterised by the BBBA05 \cite{bbba05} model and a dipole form factor with \quickmath{M_{A}^{QE} = 1.21\text{GeV}} fit to external data \cite{Aguilar_Arevalo_2010}, respectively. QE interactions only account for single-nucleon interactions whereas multi-nucleon interactions (or MEC) can contribute significantly to the overall cross-section. NEUT implements the Valencia \cite{nieves2} model to simulate MEC events, where two nucleons and two holes in the nuclear target are produced (Often called 2p2h interactions due to this effect).

For neutrinos of energy \quickmath{O(1)\text{GeV}}, PROD interactions become dominant. These predominantly produce charged and neutral pions although \quickmath{\gamma}, kaon and \quickmath{\eta} production is also considered. To simulate these interactions, the Berger-Sehgal \cite{PhysRevD.76.113004} model is implemented within NEUT. It simulates the excitation of a nucleon from a neutrino interaction, production of an intermediate baryon, and the consequential decay to a single meson or \quickmath{\gamma}. Pions can also be produced through COH interactions, which occur when the incoming neutrino interacts with the entire oxygen nuclei target leaving a single pion outside of the nucleus. NEUT utilises the Berger-Sehgal \cite{Berger_Sehgal_coh} model to simulate these interactions.

DIS and multi-\quickmath{\pi} producing interactions become the most dominant for energies \quickmath{>O(5)\text{GeV}}. PYTHIA \cite{Sjstrand1994} is used to simulate any interaction with invariant mass, \quickmath{W > 2\text{GeV/c}^{2}}, which produces at least one meson. For any interaction which produces at least two mesons but has \quickmath{W < 2\text{GeV/c}^{2}}, the Bronner model is invoked \cite{Bronner2016}. Both of these models use Parton distribution functions based on the Bodek-Yang model \cite{Gl_ck_1998,10.48550/arxiv.1011.6592,10.48550/arxiv.1012.0261}. 

\begin{figure}[h]
  \begin{subfigure}[t]{0.8\textwidth}
    \includegraphics[width=\textwidth, trim={0mm 0mm 0mm 0mm}, clip,page=1]{Figures/Selections/FSIDiagram.pdf}
  \end{subfigure}
  \caption{Illustration of the various processes which a pion can undergo before exiting the nucleus. Taken from \cite{10.48550/arxiv.1602.05299}}
  \label{fig:Selection_FSIDiagram}
\end{figure}

Any pion which is produced within the nucleus can re-interact through final state interactions before it exits, as illustrated by the scattering, absorption, production and exchange interactions in \autoref{fig:Selection_FSIDiagram}. These re-interactions alter the observable particles within the detector. For instance, if the charged pion from a CC PROD interaction is absorbed, the observables would mimic a CC QE interaction. To simulate these effects, NEUT uses a semi-classical intranuclear cascade model \cite{Hayato2021}. This cascade functions by stepping the pion through the nucleus in fixed-length steps equivalent to \quickmath{dx = R_{N}/100}, where \quickmath{R_{N}} is the radius of the nucleus. At each step, the Monte Carlo allows the pion to interact through scattering, charged exchange, absorption or production with an interaction-dependent probability calculated from a fit to external data \cite{PhysRevD.99.052007}. This cascade continues until the pion is absorbed or exits the nucleus.

Once the outgoing particle kinematics have been determined from NEUT, they are passed into the detector simulation. The near detectors ND280 and INGRID are simulated using a \texttt{GEANT4} package \cite{t2k_det,geant4} to simulate the detector geometry and particle tracking. The response of the detectors is simulated using the elecSim package. The far detector simulation is based upon the original Kamiokande experiment software which uses the \texttt{GEANT3}-based SKDETSIM \cite{Brun:1987ma,t2k_det} package. This controls the interactions of particles in the water as well as Cherenkov light production. The water quality and PMT calibration measurements detailed in \autoref{subsec:T2KSKExp_SKCalibration} are also used within this simulation to make accurate predictions of the detector response.

\section{Event Reconstruction}
\label{sec:Simulation_Reconstruction}

Any above Cherenkov threshold event which occurs in SK will be recorded by the PMT array, where each PMT records the time and accumulated charge it measured. This is shown in the event displays illustrated in \autoref{fig:Selection_SKEventDisplays}. To be useful for physics analyses, this series of PMT hit information needs to be reconstructed to determine the particle's identity and kinematics. This is because the charge and timing distribution of photons generated by a particular particle in an event is dependent upon its initial vertex position, time, direction and momentum of the particle. 

For the purposes of this analysis, the \fq reconstruction algorithm is utilised. Its core function is to compare a prediction of the accumulated charge and timing distribution from each PMT, generated for a particular particle hypothesis, to that observed in the neutrino event. It determines the best particle hypothesis by minimising a likelihood function which includes information from PMTs which were hit and those that were not hit. This improves upon the \texttt{APFit} reconstruction algorithm which has been used for many previous SK analyses. \texttt{APFit} only includes information from PMTs within the \quickmath{43\deg} Cherenkov cone and then sequentially fits the kinematic parameters and particle configuration. Conversely, \fq performs a simultaneous fit, improving both the accuracy of the fit parameters and the rejection of neutral current \quickmath{\pi^{0}} events \finish{10.1103/PhysRevLett.121.171802, 10.1103/PhysRevD.91.072010}. The \fq algorithm is based on the key concepts on the MiniBooNE reconstruction algorithm \finish{X. Li's thesis, pg 57} and described in \finish{TN146} which is summarised below.

\begin{figure}[h]
  \begin{subfigure}[t]{0.5\textwidth}
    \includegraphics[width=\textwidth, trim={0mm 0mm 0mm 0mm}, clip,page=1]{Figures/Selections/NuECandidate.pdf}
    \subcaption{\quickmath{\nu_{e}}}
  \end{subfigure}%
  \begin{subfigure}[t]{0.5\textwidth}
    \includegraphics[width=\textwidth, trim={0mm 0mm 0mm 0mm}, clip,page=1]{Figures/Selections/NuMuCandidate.pdf}
    \subcaption{\quickmath{\nu_{\mu}}}
  \end{subfigure}
  \caption{Taken from \finish{TN219}}
  \label{fig:Selection_SKEventDisplays}
\end{figure}

An event in SK can consist of multiple ``sub-events''. For example, a muon neutrino interaction will generate a muon which will subsequently decay to an electron. Both the muon and electron can generate Cherenkov photons but both subevents need to be reconstructed separately. Therefore, to avoid assigning photons generated by the decay-electron to the muon, each event is divided into time clusters, termed ``subevents'', where subevent is defined to contain at most one hit for each PMT. To find the subevents, a vertex goodness metrix is calculated for some vertex position \quickmath{\vec{x}} and time \quickmath{t},

\begin{equation}
  G(\vec{x},t) = \sum^{\texttt{hit PMTs}}_{i} \exp \left( - \frac{1}{2} \left( \frac{T_{Res}^{i}(\vec{x},t)}{\sigma} \right)^{2} \right)
\end{equation}

where

\begin{equation}
  T_{Res}^{i}(\vec{x},t) = t_{i} - t - \left| R^{i}_{PMT} - \vec{x} \right|/c_{n}
\end{equation}

is the residual hit time, \quickmath{R^{i}_{PMT}} is the position of the \quickmath{i^{th}} PMT, \quickmath{c_{n}} is the speed of light in water and \quickmath{\sigma = 4\text{ns}} \finish{Why?}. When the fit values of time and vertex are close to the true values, \quickmath{T_{Res}^{i}(\vec{x},t)} tends to zero resulting in subevents appearing as spikes in the goodness metric. The fit vertex and time are grid-scanned, and the values which maximise the goodness metrix are selected as the ``pre-fit vertex''. Whilst this predicts a vertex for use in the clustering algorithm, the final vertex is fit using the higher-precision maximum likelihood method described below.

Once the pre-fit vertex has been determined, the goodness metric is scanned as a function of \quickmath{t} to determine the number of delayed peaks. A peak-finding algorithm is then used on the goodness metric, requiring the goodness metrix to exceed some threshold and drop below a reduced threshold before any delayed additional peaks are considered. The thresholds are set such that the rate of false peak finding is minimised while still attaining good data to Monte-Carlo agreement. To improve performance, the pre-fit vertex for each delayed subevent is re-calculated after PMT hits from the primary subevent are masked. This improves the decay-electron tagging performance. Once all subevents have been determined, the time window around each subevent is then defined by the earliest and latest time which satisfies \quickmath{-180 < T_{Res}^{i} < 800 \text{ns}}. The subevents and associated time windows are then used as seeds for further reconstruction.

For a given subevent, \fq constructs a likelihood based on the accumulated charge \quickmath{q_{i}} and time information \quickmath{t_{i}} from the \quickmath{i^{th}} PMT,

\begin{equation}
  L(\Gamma, \vec{\theta}) = \prod^{\text{unhit}}_{j} P_{j}(\text{unhit}|\Gamma,\vec{\theta}) \prod^{\text{hit}}_{i} \{ 1 - P_{i}(\text{unhit}|\Gamma,\vec{\theta}) \}   f_{q}(q_{i} | \Gamma, \vec{\theta}) f_{t}(t_{i} | \Gamma, \vec{\theta}),
\end{equation}

where \quickmath{\vec{\theta}} defines the track parameters; vertex position, direction vector and momenta, and \quickmath{\Gamma} represents the particle hypothesis. \quickmath{P_{i}(\text{unhit}|\Gamma,\vec{\theta})} defines the probability of the \quickmath{i^{th}} tube to not register a hit given the track parameters and particle hypothesis. The charge likelihood, \quickmath{f_{q}(q_{i} | \Gamma, \vec{\theta})}, and time likelihood, \quickmath{f_{t}(t_{i} | \Gamma,\vec{\theta})}, respresent the probability density function of observing charge \quickmath{q_{i}} and time \quickmath{t_{i}} on the \quickmath{i^{th}} PMT given track parameters \quickmath{\vec{\theta}} and particle hypothesis \quickmath{\Gamma}.

As the generation and propagation of the optical photons is independent of the PMT and electronics response, it is natural to split the calculation into two. Firstly, calculating the expected number of photoelectrons (or predicted charge), \quickmath{\mu_{i}}, at the \quickmath{i^{th}} PMT, and then calculating the likelihood based on this value. This substitution allows the charge likelihood density \quickmath{f_{q}(q_{i} | \mu_{i})} and unhit probability \quickmath{P_{i}(\text{unhit}|\mu_{i})} to be expressed via quantities that are only dependent on the response of the PMT. 

The predicted charge is calculated based on contributions from both the direct light and the scattered light. The direct light contribution is determined based on the integration of the Cherenkov photon profile along the track. PMT angular acceptance and water quality and calibratiion measurements discussed in \autoref{subsec:T2KSKExp_SKCalibration} are included to accurately model the detector's response. The scattered light is calculated in a similar way although it includes a scattering function which depends on vertex of the particle and the position of the PMT. The charge likelihood is calculated by comparing the prediction to the observed charge in the PMT, where the prediction assumes photoelectron generation obeys a Poisson distribution.

The time likelihood is approximated to depend on the vertex \quickmath{\vec{x}}, direction \quickmath{\vec{d}}, and time \quickmath{t} of the track parameters as well as the particle hypothesis. The expected time for PMT hits is calculated by assuming unscattered photons being emitted from the midpoint of the track, \quickmath{S_{mid}},

\begin{equation}
  t^{exp}_{i} = t + S_{mid}/c + |R_{PMT}^{i} - \vec{x} - S_{mid}\vec{d}|/c_{n},
\end{equation}

where \quickmath{c} is the speed of light in vacuum. The time likelihood is then expressed in terms of the residual difference between the PMT hit time and the expected hit time, \quickmath{t^{Res}_{i} = t_{i} - t^{exp}_{i}}. As the first photon hit defines the PMT hit time, the time likelihood density profile is narrower for higher momenta particles which introduces a dependence on the predicted charge. The particle hypothesis and momentum also effect the Cherenkov photon distribution which modifies the shape of the time likelihood density since in reality not all photons are emitted at the midpoint of the track. As with the charge likelihood, the contributions from both the direct and scattered light to the time likelihood density are calculated sparately, which are both calculated from particle gun studies.

The track parameters, \quickmath{\vec{\theta}}, which maximise \quickmath{L(\Gamma | \vec{\theta})} are defined the best fit parameters. In practice MINUIT \finish{Miao's thesis, pg 75} is used to minimise the value of \quickmath{-\ln L(\Gamma, \vec{\theta})}. The particle hypothesis is determined by the comparison of  \quickmath{L(\Gamma , \vec{\theta})} across all viable hypotheses, \quickmath{\Gamma}. The fit considers an electron-like, muon-like and charged pion-like hypothesis. The particle's identity is determined by taking the ratio of the liklelihoods of each of the hypotheses. For instance, electrons and muons are distinguished by considering the value of \quickmath{\ln (L_{e}/L_{\mu})} as illustrated in \finish{Need a PID likelihood distribution here}. 

Alongside the three hypotheses which have a single final state particle generating optical photons, denoth ``single-ring'' particle hypotheses. The \fq algorithm also considers a \quickmath{\pi^{0}} hypothesis. To do this, it performs a fit looking for two standard electron-hypothesis tracks which point to the same vertex position and time. This assumes the electron tracks are generated from photon-conversion so the electron tracks actually appear offset from the proposed \quickmath{\pi^{0}} vertex. For these fits, the conversion length, direction and momenta of each photon are also considered as track parameters which are then fit in the same methodoloy as the standard single-ring hypotheses. 

The previous discussion pertains to a single final state particle which generates optical photons. However, the higher energy atmospheric neutrino events can generate finals states with multiple particles which generate Cherenkov photons. These ``multi-ring'' hypotheses are also considered in the \fq algorithm, but only for the first subevent in each ring to reduce computational cost. When calculating the charge likelihood density, the predicted charge associated with each ring is calculated separately and then merged to calculate the total accumulated charge on each PMT. Similarly, the time likelihood for the multi-ring hypothesis is calculated assuming each ring is independent. However, each track is then time ordered based on the time-of-flight from the center of the track to the PMT, and the direct light from any ring is inicident on the PMT arrives before any scattered light. To reduce computational resources required for a fit, the multi-ring fits only consider electron-like and charged pion-like rings as the pion fit can be used as a proxy for a muon fit due to their similar mass.

Typically, multi-ring fits have the largest likelihood because of the additional degrees of freedom introduced. Multi-ring fits proceed by proposing another ring to the previous fit and then fitting the parameters in the method described above. The additional ring is only added if the ratio of likelihoods between the \quickmath{n} and \quickmath{n+1} passes a criteria. The criteria values for single-ring and multi-ring separation have been determined to be \quickmath{9.35}(\quickmath{11.83}) based on Monte-Carlo studies, for hypotheses with electron-like(muon-like) first ring.

\section{Event Selection}
\label{sec:Selections_Selection}
