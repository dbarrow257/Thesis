\chapter{Conclusions and Outlook}

This thesis has presented the sensitivities of a joint beam and atmospheric neutrino oscillation analysis from the Tokai-to-Kamioka (T2K) and Super-Kamiokande (SK) experiments combining the two independent analyses presented by the collaborations \cite{Dunne2020-uf,Jiang2019-iw}. This study uses \quickmath{3244.4} days of SK livetime and \quickmath{1.97 \times 10^{21}}(\quickmath{1.63 \times 10^{21}}) POT recorded at the far detector in the neutrino(antineutrino) beam operating mode. The ND280 near detector is used within this analysis to constrain the beam flux and cross-section systematics. It uses \quickmath{1.15 \times 10^{21}}POT and \quickmath{8.34 \times 10^{20}}POT in the neutrino and antineutrino running modes, respectively.

%These constraints are applied to both the beam far detector and low energy atmospheric samples through a correlated neutrino interaction model.
%This ensures that a consistent interaction model is used throughout the analysis. This is the first example of applying the T2K near detector constraints onto the SK atmospheric samples inside an oscillation analysis.

This analysis uses a Bayesian Markov Chain Monte Carlo fitting technique implemented within the \texttt{MaCh3} framework. This analysis has significantly developed the fitting framework, both in terms of technical features and performance. This includes supporting new samples, systematics, and oscillation channels. These developments have become the foundation of the fitter's expansion into other neutrino oscillation experiments. Beyond these improvements, a novel technique for calculating the atmospheric neutrino oscillation probabilities has been developed. This calculation uses a sub-sampling linear-averaging approach to ensure that the sensitivities being calculated are not biased due to insufficient Monte Carlo statistics in a region of rapidly varying probability. It illustrates a computationally feasible method of reliably calculating oscillation probabilities that can be utilised within any fitting framework.

%As a requirement for this analysis, the predominantely T2K fitting framework was required to simultaneously support and reweight alternative Monte Carlo samples. These developments include supporting new systematics, new oscillation probability calculations and previously unconsidered oscillation channels that saw the first tau events incorporated into the fitter. The developments required to realise this analysis have been the building blocks of the frameworks expansion into other experiments. 
%Due to the MCMC techniques used within the fit,

%Further techniques, which consider uncertainties related to modelling the interaction height of the primary cosmic rays and the density of the Earth have also been implemented. Whilst of critical importance to this analysis, these methods are a stand-alone technique and can be used in any other atmospheric oscillation analysis. Alongside these physics considerations, an alternative oscillation calculation engine has been interfaced with the framework to significantly reduce the resources required to perform this analysis. Current developments, based on the benefits illustrated within this analysis, are being considered within the T2K analysis.

The sensitivity of the joint beam-atmospheric analysis is presented in \autoref{tab:Conclusion_Summary}, and compared to the beam-only analysis \cite{Dunne2020-uf}. The sensitivities are evaluated using a set of known oscillation parameter values close to the results from a previous T2K analysis \cite{PhysRevLett.112.181801} (denoted AsimovA in \autoref{tab:Conclusion_Summary}). The joint analysis has a stronger sensitivity to \quickmath{\sin^{2}(\theta_{23})}, as evidenced by the tighter \quickmath{1\sigma} credible intervals when the constraints from reactor experiments are not applied. The joint fit's sensitivity to \quickmath{\delta_{CP}} is marginally stronger than beam-only analysis but would not change any conclusion which would be made. Whilst the sensitivity to \quickmath{|\Delta m^{2}_{32}|} is mostly unchanged between the two analyses, the sensitivity to select the correct hierarchy given is significantly improved. This follows from a substantial preference for the normal hierarchy hypothesis presented within the joint analysis, as classified by Jeffrey's scale \cite{Jeffreys:1939xee}. This is notable as the beam-only analysis can not make this statement, either with or without the application of the reactor constraint. The joint fit's preference for the correct hierarchy increases once the reactor constraint is applied. The preference for selecting the correct octant of \quickmath{\sin^{2}(\theta_{23})} is classified as weak by Jeffrey's scale but is still stronger than the statement made by the beam-only analysis.

\begin{table}[ht!]
  \centering
  \begingroup
  \renewcommand{\arraystretch}{1.5}
  \resizebox{\textwidth}{!}{
    \begin{tabular}{|l|c|c|c|c|c|}
      \hline
      Fit & \quickmath{\delta_{CP}} (HPD) & \quickmath{\Delta m^{2}_{32}}              & \quickmath{\sin^{2}(\theta_{23})} & \quickmath{B(\text{NH}/\text{IH})} & \quickmath{B(\text{UO}/\text{LO})} \\
      &                               & \quickmath{[\times 10^{-3} \text{eV}^{2}]} &                                   &                                    & \\
      \hline
      \hline
      Asimov A & \quickmath{-1.601} & \quickmath{2.509} & \quickmath{0.528} & NH & UO \\
      \hline
      Beam & \quickmath{-1.45^{+0.06}_{-0.06}} & \quickmath{2.51^{+0.07}_{-0.06}} & \quickmath{0.501^{+0.044}_{-0.026}} & \quickmath{1.91} & \quickmath{1.56} \\
      Beam w/RC & \quickmath{-1.57^{+0.06}_{-0.06}} & \quickmath{2.51^{+0.08}_{-0.06}} & \quickmath{0.533^{+0.022}_{-0.043}} & \quickmath{3.09} & \quickmath{2.47} \\
      Joint & \quickmath{-1.57^{+0.06}_{-0.06}} & \quickmath{2.51^{+0.07}_{-0.06}} & \quickmath{0.518^{+0.027}_{-0.038}} & \quickmath{3.67} & \quickmath{1.74} \\
      Joint w/RC & \quickmath{-1.57^{+0.06}_{-0.06}} & \quickmath{2.51^{+0.05}_{-0.06}} & \quickmath{0.528^{+0.027}_{-0.038}} & \quickmath{6.47} & \quickmath{2.64} \\
      \hline
      \hline
    \end{tabular}
  }
  \caption{}
  \label{tab:Conclusion_Summary}
  \endgroup
\end{table}

The sensitivities of the beam-only and joint atmospheric-beam fit have also been compared at a set of known oscillation parameters which are CP-conserving and in the lower octant of \quickmath{\sin^{2}(\theta_{23})}. The joint analysis has an \quickmath{\sim 5\%} improved ability to select the known values more precisely compared to the beam-only analysis. This is further evidenced by the larger Bayes factor for preferring the correct hierarchy and octant hypothesis. 

Whilst this analysis provides the first sensitivities of a joint beam and atmospheric analysis, there are more improvements that could be made. Since this analysis began, T2K has released an updated oscillation analysis with additional near and far detector samples alongside a more sophisticated interaction model \cite{Bronner2022-wd}. The overall change in oscillation parameter measurement observed by T2K was relatively minor but the stronger constraints on the systematics could impact this joint analysis to a larger extent.
%This, or a particular focus on CCRES interacton modelling, could lead to a better understanding of the CC\quickmath{1\pi} samples from a physics-driven perspective rather than invoking the ad-hoc systematic used in this analysis.
Further developments could consider the effect of correlating the beam and atmospheric flux uncertainties, where updates of the Bartol and Honda models may allow this to be studied \cite{Sato2022-ss}.
%The next goal for this analysis would be moving to a data fit. This would require performing studies which aim to understand the effect of the model choice on the oscillation parameter measurements. This tests whether there is freedom in the systematics model to allow alternative models to be fit therefore resulting in more reliable measurements. 

Beyond these model improvements, more data is available than what is assumed for this analysis. The T2K experiment has accumulated an additional \quickmath{1.78 \times 10^{20}}POT in neutrino mode. Similarly, there are several early SK periods that have not been considered within this analysis as the reconstruction software used in this analysis has not been validated for those periods. The SK-Gd era will also continue to accumulate statistics. Developments in the atmospheric sample selections may also benefit from the Gadolinium dopants as neutron capture will aid in neutrino/antineutrino separation leading to better mass hierarchy sensitivity. This would require including interaction systematics for neutron capture of Gadolinium which has already started \cite{10.48550/arxiv.2209.08609}.

This analysis shows the increased sensitivity to oscillation parameters from the combination of beam and atmospheric samples. It has developed the \texttt{MaCh3} fitting framework and has laid the foundations of the fitter's expansion into other neutrino oscillation experiments. The sensitivities provided in this analysis, and the techniques which were used to generate them, are significant to the future of neutrino oscillation physics which will likely perform similar analyses. As such, they have been presented by the T2K and SK collaborations at the Neutrino 2022 conference \cite{Bronner2022-wd}. Moving towards the next generation of neutrino experiments, this analysis has the potential to become the basis of the oscillation analysis for future Hyper-Kamiokande experiments.

%. SK-I to SK-III contained approximately the same number of statistics as the SK-IV period used within this analysis but were neglected as the \texttt{fiTQun} reconstruction algorithm has not validated for those periods. Furthermore, the SK-V era provides an additional \quickmath{583} days of data-taking which could also be included within this analysis. This would require updating the detector systematics to reflect the changes brought about by the detector refurbishment.

%The near detector of the T2K experiment is currently undergoing development work to include new components. This lowers the required energy thresholds and improves vertex resolution. This may lead to stronger constraints on the flux and cross-section systematics used within the beam analysis, which could strengthen the sensitivities provided within this analysis. 

