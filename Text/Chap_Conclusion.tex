\chapter{Conclusions and Outlook}

This thesis has presented the sensitivites of the first offcial joint beam and atmospheric neutrino oscillation parameter measurements from the Tokai-to-Kamioka (T2K) and Super-Kamiokande (SK) collaborations. It combines the two independent analyses presented by the two independent collaborations \finish{citations}. This equates to a combined \quickmath{3244.4} days equivalent of SK-IV livetime and \quickmath{1.97 \times 10^{21}}(\quickmath{1.63 \times 10^{21}}) POT in the neutrino(antineutrino) beam operating mode.

The ND280 near detector is used to constrain the flux and cross-section systematics envoked within this oscillation analysis. It uses \quickmath{1.15 \times 10^{21}}POT and \quickmath{8.34 \times 10^{20}}POT in the neutrino and antineutrino running modes, respectively. These constraints are applied to both the beam far detector and low energy atmospheric samples through a correlated neutrino interaction model. This ensures that a consistent interaction model is used throughout the analysis. This is the first example of applying the T2K near detector constraints onto the SK atmospheric samples inside an oscillation analysis. \finish{More physics developments}

This analysis implements a Bayesian Markov Chain Monte Carlo (MCMC) fitting technique built within the \texttt{MaCh3} framework. This analysis has significantly developed the fitting framework, both in terms of technical features and performance. As a requirement for this analysis, the predominantely T2K fitting framework was required to simultaneously support and reweight alternative Monte Carlo samples. These developments include supporting new systematics, new oscillation probability calculations and previously unconsidered oscillation channels that saw the first tau events incorporated into the fitter. The developments required to realise this analysis have been the building blocks of the frameworks expansion into other experiments. 

Due to the MCMC techniques used within the fit, a novel technique of calculating the atmospheric neutrino oscillation probabilities has been developed. This calculation uses a sub-sampling linear-averaging approach to ensure that the sensitivities being calculated are not biased due to insufficient Monte Carlo statistics in a region of fast varying probability. It illustrates a computationally feasible method of reliably calculating oscillation probabilities which can be utilised within any fitting framework. Further techniques, which considre uncertainties related to modelling the interaction height of the primary cosmic rays and the density of the Earth have also been implemented. Whilst of critical importance to this analysis, these techniques are a stand-alone technique and can be used in any other atmospheric oscillation analysis. Alongside these physics considerations, an alternative oscillation calculation engine has been interfaced with the framework to significantly reduce the resources required to perform this analysis. Current developments, based on the benefits illustrated within this analysis, are being considered within the T2K analysis.

\finish{Physics results here}

Whilst this analysis provides the first sensitivity measurement of a joint beam and atmospheric analysis, there are more improvements to be made. Since this analysis began, T2K has released an updated oscillation analysis with additional near and far detector samples alongside a more sophisticated interaction model. The overall change in oscillation parameter measurement observed by T2K was relatively minor \finish{Bronner Nu2022} but the stronger constraints on the systematics could impact this joint analysis to a larger extent. This, or a particular focus on CCRES interacton modelling, could lead to a better understanding of the CC\quickmath{1\pi} samples from a physics-driven perspective rather than invoking the ad-hoc systematic used in this analysis. Further developments should consider the effect of correlating the beam and atmospheric flux uncertainties, where updates of the Bartol and Honda models are being made to realise this. The next goal for this analysis would be moving to a data fit. This would require performing studies which aim to understand the effect of the model choice on the oscllation parameter measurements. This tests whether there is freedom in the systematics model to allow alternative models to be fit therefore resulting in more reliable measurements. 

Beyond these model improvements, more data is available than what is assumed for this analysis. The T2K experiment has ran an additional period of two months, corresponding to an additional \quickmath{1.78 \times 10^{20}}POT in neutrino mode. Similarly, there are several SK periods which have not been considered within this analysis. SK-I to SK-III contained approximately the same number of statistics as the SK-IV period used within this analysis but were neglected as the \texttt{fiTQun} reconstruction algorithm has not validated for those periods. Furthermore, the SK-V era provides an additional \quickmath{583} days of data-taking which could also be included within this analysis. This would require updating the detector systematics to reflect the changes brought about by the detector refurbishment.

The T2K and SK experiments are continually developing. The near detector of the T2K experiment is currently undergoing development work to include new components. This lowers the required energy thresholds and improves vertex resolution. This may lead to stronger constraints on the flux and cross-section systematics used within the beam analysis, which could strengthen the sensitivities provided within this analysis. The SK-Gd era will also continue to accumulate statisitics. Developments in the atmospheric sample selections may also benefit from the Gadolinium dopants as neutron capture will aid in neutrino/antineutrino separation leading to better mass hierarchy sensitivity. This would require including interaction systematics for neutron capture of Gadolinium which has already started \finish{Citation to SK neutron paper: Arxiv - 2209.08609}.

This analysis presents the sensitivities of the first joint beam and atmospheric analysis. This analysis and the supporting framework has the potential to become the basis of the oscillation analysis for future Hyper-Kamiokande experiment.

\begin{itemize}
\item Predicted number of events at the FD
\item SK only results - w/wo RC
\item Bayes factor
\item Joint fit has significant preference for correct hierarchy without external constraints - T2K doesn't have that sensitivity
\item Summary table
\item Published at Nu2022
  
\item Further development of correlated detector model

\end{itemize}
