\chapter{Introduction}
\label{chap:Introduction}

The Super-Kamiokande experiment is situated as the far detector of the T2K experiment and observes neutrinos from both the beam originating in J-PARC alongside the flux of atmospheric neutrinos emitted from the primary and secondary interactions of cosmic rays. Previous oscillation analyses officially supported by the two experiments have been independent of one another. However, due to the different energies, path-lengths and density of matter in which the neutrinos pass through, a combined analysis will be able to leverage the constraints from both experiments and be able to break some of degeneracies in oscillation parameter space.

This thesis details the sensitivities of a joint beam and atmospheric neutrino analysis using beam samples detected at the near and far detectors of the T2K experiments and atmospheric samples measured at SK. It combines the beam analysis presented in \cite{Dunne2020-uf} and the atmospheric analysis documented in \cite{Jiang2019-iw}. This corresponds to run1-10 of the T2K experiment with approximately equal data taken in neutrino and antineutrino beam modes, alongside more than \quickmath{3000} days of atmospheric data. This analysis will have sensitivity to the \quickmath{\delta_{CP}}, \quickmath{\sin^{2}(\theta_{13})}, \quickmath{\sin^{2}(\theta_{23})}, and \quickmath{\Delta m^{2}_{32}} oscillation parameters. Crucially, the combination of beam and atmospheric neutrinos should give strong sensitivity to the mass hierarchy hypothesis due to the correlation between the matter resonance and \quickmath{\sin^{2}(\theta_{23})}.

Chapter 2 provides a concise overview of neutrino physics history including the discovery of the neutrino, the first evidence for neutrino oscillation and 
