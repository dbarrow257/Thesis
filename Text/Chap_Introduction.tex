\chapter{Introduction}
\label{chap:Introduction}

Current astrophysical measurements show that the universe is matter-dominated, despite current theoretical models suggesting that an equal amount of matter and antimatter were created in the Big Bang.
For an imbalance to occur, the Sakharov conditions \cite{Sakharov1991} require the violation of charge parity (CP) symmetries.
%One explanation of this behaviour is through the violation of charge parity (CP) symmetries, one requirement of the Sakharov conditions \cite{Sakharov1991}.
%, which if violated, could result in the observed matter-antimatter imbalance.
CP violation has been observed in quark mixing but is insufficient to explain the observed asymmetry.
As the Standard Model relates the neutrino and the antineutrino through these symmetries, CP violation could be found in the neutrino sector.
This would be observed as a difference between neutrino and antineutrino oscillation.
%through the \quickmath{\delta_{CP}} phase contained within the PMNS formalism.
%Neutrino oscillation physics has the potential to include CP-violating terms 
Current neutrino oscillation measurements contain hints of CP violation \cite{Dunne2020-uf} but no conclusive measurement has been achieved.
One of the main goals of neutrino oscillation experiments is to continue the search for CP violation, requiring a precise measurement of all oscillation parameters.
%This requires a precise measurement of all oscillation parameters including the currently undetermined neutrino mass hierarchy.
%The next generation of neutrino experiments will combine information from both beam and atmospheric neutrino sources for consistent measurements of the mixing parameters, mass hierarchy and CP-violating phase present within neutrino oscillations.

%The Super-Kamiokande (SK) detector is situated as the far detector of the Tokai-to-Kamioka (T2K) experiment and observes neutrinos from a beam originating in J-PARC alongside the interactions of atmospheric neutrinos emitted from the subsequent decays of cosmic rays.
The Super-Kamiokande (SK) detector observes atmospheric neutrinos emitted from the subsequent decays of cosmic rays. It is also situated as the far detector of the Tokai-to-Kamioka (T2K) experiment and measures the interactions of neutrinos produced from the J-PARC (anti-)neutrino beam facility. Previous oscillation analyses published by these two experiments have treated the datasets independently. However, due to the different energies, path lengths, and density of matter in which the neutrinos pass through, a combined analysis can leverage the constraints from both datasets and may be able to improve measurements of neutrino oscillation parameters.
%break some of the degeneracies in oscillation parameter space.

This thesis introduces a joint beam and atmospheric neutrino analysis using beam samples observed at the near and far detectors of the T2K experiment and atmospheric samples present in SK. It combines the beam analysis presented in \cite{Dunne2020-uf} and the atmospheric analysis documented in \cite{Jiang2019-iw}. This corresponds to run1-10 of the T2K experiment with approximately equal exposure taken in neutrino and antineutrino beam modes, alongside more than \quickmath{3000} days of atmospheric events. The results in this thesis are presented as sensitivities to the \quickmath{\delta_{CP}}, \quickmath{\sin^{2}(\theta_{13})}, \quickmath{\sin^{2}(\theta_{23})}, and \quickmath{\Delta m^{2}_{32}} oscillation parameters. Crucially, the combination of beam and atmospheric neutrinos gives strong sensitivity to the mass hierarchy due to the correlation between the matter resonance and \quickmath{\sin^{2}(\theta_{23})}. The sensitivities are generated by the \texttt{MaCh3} Bayesian Markov Chain Monte Carlo fitting framework.
%This analysis lays the foundation of a joint analysis in Hyper-Kamiokande, which is one of the next-generation neutrino oscillation experiments.

Chapter 2 provides a concise overview of neutrino physics history including the discovery of the neutrino along with the first evidence for neutrino oscillation. It also includes a brief discussion of the theory underpinning the PMNS formalism of neutrino oscillation alongside a summary of the current measurements of each oscillation parameter.
%Furthermore, a description of how beam and atmospheric neutrino experiments are sensitivity to each oscillation parameter is provided.

The T2K and SK experiments are detailed in Chapter 3. This includes the design and calibration of the SK detector along with a brief description of the composition and detection techniques of T2K's two near detectors. The neutrino beamline, and the `off-axis' technique, are also briefly summarised. 

This thesis presents a Bayesian neutrino oscillation analysis that uses Markov Chain Monte Carlo techniques. This analysis strategy, along with a summary of the fundamental concepts of Bayesian inference, is described in Chapter 4. This includes a discussion about the conditions that are required to correctly sample the parameter space along with the methods used to calculate parameter estimations and build credible intervals.

Chapter 5 details the simulations and reconstruction tools used to build Monte Carlo predictions of each sample used within this analysis. This includes the models used to provide flux predictions of the beam and atmospheric neutrinos as well as the models invoked with this analysis to simulate neutrino interactions.
%Validation of the far detector's reconstruction tools has been documented which compares the change in detector response between two different periods of SK.

A description of the beam samples used at the near and far detector and the atmospheric samples used at the SK detector is presented in Chapter 6. These include energy and interaction mode comparisons along with documenting the event selection cuts. This chapter also describes the systematic models used to quantify the uncertainty in the flux predictions of both beam and atmospheric neutrinos, the interaction models, and the response of the detectors used within this analysis.

A novel atmospheric neutrino oscillation probability calculation method is documented in Chapter 7. This is required to ensure reliable Monte Carlo sampling of a rapidly varying region of oscillation parameter space. This chapter also documents the uncertainties related to the Earth's density as well as the production height of neutrinos in the upper atmosphere.

Chapter 8 presents the sensitivities of this joint beam and atmospheric neutrino oscillation analysis.
%This utilises the run1-10 T2K statistics and more than \quickmath{3000} days equivalent of atmospheric events.
The results are provided for two different sets of known values. The application of the external constraints on \quickmath{\sin^{2}(\theta_{13})} has also been considered. The sensitivities of the joint analysis are compared to the beam-only analysis and show the benefits of the combined analysis. These results have been presented by the T2K and SK collaborations at the Neutrino 2022 conference \cite{Bronner2022-wd}.

A summarised discussion of the sensitivity results and the outlook for the analysis, including the implications of this analysis on the next generation of neutrino experiments, is provided in Chapter 9.
