\begin{center}
  \huge{Statement of Originality}
\end{center}

The work, and associated interpretation, presented within this thesis are my own and were produced by myself. Where applicable, results and figures taken from others have been attributed and referenced. This work has not been submitted for any other qualification, at this university or any other.

The background of neutrino physics history and a description of the T2K and SK experiments are provided in Chapters 2 and 3, respectively. The Bayesian fitting techniques and \texttt{MaCh3} framework used throughout this analysis are detailed in Chapter 4. These chapters present entirely background knowledge relevant for understanding the work presented within this thesis. Chapter 5 documents the simulation and reconstruction of neutrino events. This includes a section of work done by myself to validate the \texttt{fiTQun} reconstruction software for use on SK-V data.

Chapter 6 details the event selections and systematics used within this oscillation analysis. The selections were developed by others within the T2K and SK collaborations and have been appropriately referenced. The implementation and validation of the simultaneous support for the beam and atmospheric samples, selected by the SK detector, in the \texttt{MaCh3} fitting framework has been entirely my own work. The systematic models invoked within this analysis were developed by others, and relevant references have been included. The implementation of these systematics within the \texttt{MaCh3} fitting framework has been entirely my own work and includes generating the systematic response functions for each systematic from the Super-Kamiokande Monte Carlo. The implementation and validation of the near detector samples and systematics were performed by Clarence Wret on behalf of the \texttt{MaCh3} working group.

Chapter 7 documents a new method of calculating the oscillation probabilities for atmospheric neutrinos at SK. The method and validation of the `smearing' technique were entirely my own. The methodology for including effects from production height systematics was developed by others but the implementation and validation were my own. Several performance increases, including the interfacing of an alternative oscillation calculation engine, were my own work. The sensitivities, and interpretation, provided within Chapter 8 are produced entirely by myself. 
