\begin{mclistof}{List of Abbreviations}{2.5cm}

%Theory
\item[CP]Charge Parity. A parameter contained within the PMNS neutrino oscillation matrix which describes the difference between neutrino and antineutrino oscillation probability.
\item[MH]Mass Hierarchy. The ordering of the second and third neutrino mass states, which is currently unknown.

\item[MSW]Mikheyev Smirnov Wolfenstein effect. If neutrinos pass through matter, the oscillation probability is modified due to the presence of electrons within the matter.
\item[CKM]Cabibbo Kobayashi Maskawa matrix. The quark mixing matrix which is analougous to the PMNS matrix.
\item[PMNS]Pontecorvo Maki Nakagawa Sakata matrix. The matrix which describes neutrino oscillation between weak flavour eigenstates and mass eigenstates.

\item[PREM]Preliminary Reference Earth Model. A model which describes the density of the Earth as a function of radius.

%Detectors
\item[T2K]Tokai-To-Kamioka experiment.
\item[SK]Super-Kamiokande experiment.

\item[ND280]Near Detector (ND) complex situated at 280m from proton target. This detector is situated \quickmath{2.5\deg} off-axis with respect to the beam direction.
\item[FD]Far Detector of the T2K experiement (SK). This detector is situated \quickmath{2.5\deg} off-axis with respect to the beam direction.

\item[ECAL]Electromagnetic Calorimeter. ND280 contains two ECALs: The P0D ECAL and the tracker ECAL which surrounds the TPCs and FGDs.
\item[FGD]Fine Grained Detector. ND280 contains two FGDs which are utilised as a target for neutrino interactions.
\item[SMRD]Side Muon Range Detector. Subdetector of the ND280 located within the magnet.
\item[TPC]Time Projection Chamber. ND280 contains three TPCs which are used for particle identity and kinematic reconstruction.

\item[ID]Inner Detector. The area inside the cylindrical support structure of the SK detector.
\item[OD]Outer Detector. The area outside the cylindrical support structure of the SK detector.

\item[MPPC]Multi Pixel Photon Counter. A device used to detect scintillation light collected within the INGRID and ND280 detectors.
\item[PMT]Photo-Multiplier Tube. A device used to detect Cherenkov light produced within the SK detector.
\item[DAQ]Data Acquisition. The process of collecting the electronic readout from all parts within a detector.

%Technique
\item[MC]Monte Carlo simulation.  
\item[MCMC]Markov Chain Monte Carlo. The technique used within this thesis to fit the oscillation parameters.
\item[Asimov]Asimov data set. A dataset built from the full MC prediction to eliminate statistical fluctuations.

%Analysis
\item[PID]Particle Identification.

\item[FHC]Forward Horn Current. T2K beam configuration which produces a beam predominantely consisting of \quickmath{\nu_\mu}.
\item[RHC]Reverse Horn Current. T2K beam configuration which produces a beam predominantely consisting of \quickmath{\bar{\nu}_\mu}.
\item[POT]Protons-On-Target. The measure of how much neutrino data has been collected by T2K.
\item[BANFF]Beam And ND280 Flux extrapolation task force. A group in T2K which performs the ND280 fit to constrain flux and cross-section systematics. 
\item[BANFF fit]The posterior constraint from the BANFF fit. Interchangably described as the `post-BANFF' or `post-ND' constraints within this analysis.

\item[FC]Fully Contained. Events within the SK detector that have no significant activity within the OD.
\item[PC]Partially Contained. Events within the SK detector that do have significant activity within the OD.
\item[MR]Multi-Ring. Events within the SK detector that contain multiple particles which create Cherenkov light.

\item[NEUT]Neutrino event generator. Software used to model neutrino interactions at both the ND and FD.
\item[SKDETSIM]Software used to model the detector response at SK
\item[APFit]Event reconstruction software used at SK. This is used within the official SK oscillation analysis.
\item[FiTQun]Event reconstruction software used at SK. This is used within the official T2K oscillation analysis.

%OA
\item[Bayes Factor]A metrix used for hypothesis testing used within Bayesian statistics. It is equal to the ratio of the likelihood for each hypothesis and provides data-driven evidence of preference for one model or the other.
\item[NH]Normal Hierarchy. The neutrino mass ordering where \quickmath{m_{3}^{2} > m_{2}^{2}}.
\item[IH]Inverted Hierarchy. The neutrino mass ordering where \quickmath{m_{3}^{2} < m_{2}^{2}}.
\item[UO]Upper Octant. The region where \quickmath{\sin^{2}(\theta_{23}) > 0.5}.
\item[LO]Lower Octant. The region where \quickmath{\sin^{2}(\theta_{23}) < 0.5}.

\item[HPD]HPD credible interval]Highest Posterior Density credible interval. A technique to define a credible interval which contains a specific fraction of the entire posterior distribution. It requires that every point within the interval has a higher posterior probability density than every point outside of the interval.
\item[Credible Interval]Method of definining the uncertainty on a point estimate in Bayesian statistics used within this thesis. A \quickmath{95\%} interval contains \quickmath{95\%} of the posterior probability.
\item[RC]Reactor Constraint. An external constraint on the \quickmath{\sin^{2}(\theta_{13})} oscillation parameter.

%Interactions
\item[QE]Quasi-Elastic interaction. An interaction where the neutrino interacts with the entire nucleon, through a 2-particle \quickmath{\rightarrow} 2-particle interaction.
\item[RES]Resonant Production interaction. An interaction which produces a single pion within the final state.
\item[DIS]Deep Inelastic Scattering interaction. An interaction where the neutrino interacts with the consistuent particles of a nucleon.
\item[MEC]Meson Exchange Current interaction. An interaction where the neutrino interacts with a mutli-nucleon state rather than a single nucleon. Interchangably termed `\texttt{2p2h}' throughout this thesis.
\item[FSI]Final State Interaction. The reinteraction of particles produced within the primary interaction within the nucleus.
\item[SI]Secondary Interaction. The reinteraction of particles which leave the primary nucleus before they are measured. SI and FSI are separated depending upon whether the reinteraction occured within the primary nucleus.
  
\item[Dial]A parameter used within the oscillation analysis which can be varied to modify the MC prediction.
\item[\quickmath{E_{rec}}]Reconstructed Neutrino Energy. A value calculated under the assumption of a CCQE or CCRES interaction on a stationary nucleon.

\item[Likelihood]The probability of observing some data given some MC prediction.
\item[Marginalisation]The method used within this analysis to integrate over nuisance parameters from the posterior distribution so parameters of interest can be studied.
\item[Nuisance parameter]A model parameter than can modify the MC prediction but is not a parameter of interest to this analysis.
\item[P0D]The \quickmath{\pi^0} Detector. A subdetector contained within the ND280.
\item[Posterior Distribution]The probability distribution for some model parameters given the data.
\item[SF]Spectral Function. A nuclear model used to simulate the ground state of a nucleus.
  
\end{mclistof} 
