%\begin{mclistof}{List of Abbreviations}{2.5cm}
\chapter*{List of Abbreviations}
\addcontentsline{toc}{chapter}{List of Abbreviations}

\begin{itemize}[label={},leftmargin=*]
  
\item \textbf{APFit}: Event reconstruction software used at SK. This is used within the official SK oscillation analysis.
\item \textbf{Asimov}: Asimov data set. A data set built from the full MC prediction to eliminate statistical fluctuations.
\item \textbf{BANFF}: Beam And ND280 Flux extrapolation task force. A group in T2K which performs the ND280 fit to constrain flux and cross-section systematics. 
\item \textbf{BANFF fit}: The posterior constraint from the BANFF fit. Interchangeably described as the `post-BANFF' or `post-ND' constraints within this analysis.
\item \textbf{Bayes Factor}: A metric used for hypothesis testing used within Bayesian statistics. It is equal to the ratio of the likelihood for each hypothesis and provides data-driven evidence of preference for one model or the other.
\item \textbf{CP}: Charge Parity. A parameter contained within the PMNS neutrino oscillation matrix which describes the difference between neutrino and antineutrino oscillation probability.
\item \textbf{Credible Interval}: Method of defining the uncertainty on a point estimate in Bayesian statistics used within this thesis. A \quickmath{95\%} interval contains \quickmath{95\%} of the posterior probability.
\item \textbf{DAQ}: Data Acquisition. The process of collecting the electronic readout from all parts within a detector.
\item \textbf{Dial}: A parameter used within the oscillation analysis which can be varied to modify the MC prediction.
\item \textbf{DIS}: Deep Inelastic Scattering interaction. An interaction where the neutrino interacts with the constituent particles of a nucleon.
\item \textbf{ECAL}: Electromagnetic Calorimeter. ND280 contains two ECALs: The P0D ECAL and the tracker ECAL which surrounds the TPCs and FGDs.
\item \textbf{\quickmath{E_{rec}}}: Reconstructed Neutrino Energy. A value calculated under the assumption of a CCQE or CCRES interaction on a stationary nucleon.
\item \textbf{FC}: Fully Contained. Events within the SK detector that have no significant activity within the OD.
\item \textbf{FiTQun}: Event reconstruction software used at SK. This is used within the official T2K oscillation analysis.
\item \textbf{FD}: Far Detector of the T2K experiment (SK). This detector is situated \quickmath{2.5\deg} off-axis with respect to the beam direction.
\item \textbf{FGD}: Fine Grained Detector. ND280 contains two FGDs which are utilised as a target for neutrino interactions.
\item \textbf{FHC}: Forward Horn Current. T2K beam configuration which produces a beam predominantly consisting of \quickmath{\nu_\mu}.
\item \textbf{FSI}: Final State Interaction. The re-interaction of particles produced within the primary interaction within the nucleus.
\item \textbf{HPD credible interval}: Highest Posterior Density credible interval. A technique to define a credible interval which contains a specific fraction of the entire posterior distribution. It requires that every point within the interval has a higher posterior probability density than every point outside of the interval.
\item \textbf{ID}: Inner Detector. The area inside the cylindrical support structure of the SK detector.
\item \textbf{IH}: Inverted Hierarchy. The neutrino mass ordering where \quickmath{m_{3}^{2} < m_{2}^{2}}.
\item \textbf{Likelihood}: The probability of observing some data given some MC prediction.
\item \textbf{LO}: Lower Octant. The region where \quickmath{\sin^{2}(\theta_{23}) < 0.5}.
\item \textbf{Marginalisation}: The method used within this analysis to integrate over nuisance parameters from the posterior distribution so parameters of interest can be studied.
\item \textbf{MC}: Monte Carlo simulation.  
\item \textbf{MCMC}: Markov Chain Monte Carlo. The technique used within this thesis to fit the oscillation parameters.
\item \textbf{MEC}: Meson Exchange Current interaction. An interaction where the neutrino interacts with a multi-nucleon state rather than a single nucleon. Interchangeably termed `\texttt{2p2h}' throughout this thesis.
\item \textbf{MH}: Mass Hierarchy. The ordering of the second and third neutrino mass states, which is currently unknown.
\item \textbf{MPPC}: Multi Pixel Photon Counter. A device used to detect scintillation light collected within the INGRID and ND280 detectors.
\item \textbf{MR}: Multi-Ring. Events within the SK detector that contain multiple particles which create Cherenkov light.
\item \textbf{MSW}: Mikheyev Smirnov Wolfenstein effect. If neutrinos pass through matter, the oscillation probability is modified due to the presence of electrons within the matter.
\item \textbf{ND280}: Near Detector (ND) complex situated at 280m from proton target. This detector is situated \quickmath{2.5\deg} off-axis with respect to the beam direction.
\item \textbf{NEUT}: Neutrino event generator. Software used to model neutrino interactions at both the ND and FD.
\item \textbf{NH}: Normal Hierarchy. The neutrino mass ordering where \quickmath{m_{3}^{2} > m_{2}^{2}}.
\item \textbf{Nuisance parameter}: A model parameter than can modify the MC prediction but is not a parameter of interest to this analysis.
\item \textbf{OD}: Outer Detector. The area outside the cylindrical support structure of the SK detector.
\item \textbf{P0D}: The \quickmath{\pi^0} Detector. A subdetector contained within the ND280.
\item \textbf{PC}: Partially Contained. Events within the SK detector that do have significant activity within the OD.
\item \textbf{PID}: Particle Identification.
\item \textbf{PMNS}: Pontecorvo Maki Nakagawa Sakata matrix. The matrix which describes neutrino oscillation between weak flavour eigenstates and mass eigenstates.
\item \textbf{PMT}: Photo-Multiplier Tube. A device used to detect Cherenkov light produced within the SK detector.
\item \textbf{Posterior Distribution}: The probability distribution for some model parameters given the data.
\item \textbf{POT}: Protons-On-Target. The measure of how much neutrino data has been collected by T2K.
\item \textbf{PREM}: Preliminary Reference Earth Model. A model which describes the density of the Earth as a function of radius.
\item \textbf{QE}: Quasi-Elastic interaction. An interaction where the neutrino interacts with the entire nucleon, through a 2-particle \quickmath{\rightarrow} 2-particle interaction.
\item \textbf{RC}: Reactor Constraint. An external constraint on the \quickmath{\sin^{2}(\theta_{13})} oscillation parameter.
\item \textbf{RES}: Resonant Production interaction. An interaction which produces a single pion within the final state.
\item \textbf{RHC}: Reverse Horn Current. T2K beam configuration which produces a beam predominantly consisting of \quickmath{\bar{\nu}_\mu}.
\item \textbf{SF}: Spectral Function. A nuclear model used to simulate the ground state of a nucleus.
\item \textbf{SK}: Super-Kamiokande experiment.
\item \textbf{SKDETSIM}: Software used to model the detector response at SK
\item \textbf{SI}: Secondary Interaction. The re-interaction of particles which leave the primary nucleus before they are measured. SI and FSI are separated depending upon whether the re-interaction occurred within the primary nucleus.
\item \textbf{SMRD}: Side Muon Range Detector. Subdetector of the ND280 located within the magnet.
\item \textbf{T2K}: Tokai-To-Kamioka experiment.
\item \textbf{TPC}: Time Projection Chamber. ND280 contains three TPCs which are used for particle identity and kinematic reconstruction.
\item \textbf{UO}: Upper Octant. The region where \quickmath{\sin^{2}(\theta_{23}) > 0.5}.

\end{itemize}

%\end{mclistof} 
