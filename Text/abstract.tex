Super-Kamiokande (SK) is a large water Cherenkov detector which observes a flux of atmospheric neutrinos originating from the primary and secondary decays of cosmic rays. It is also situated as the far detector of the Tokai-to-Kamioka (T2K) experiment, with a baseline of \quickmath{295\text{km}}, and observes the flux of beam neutrinos (or antineutrinos) produced at the J-PARC facility. This makes the detector the ideal candidate for a joint beam-atmospheric oscillation analysis.

This thesis presents the first simultaneous beam and atmospheric oscillation analysis to be supported by the two collaborations. This leverages the different energies and baselines of the two experiments and provides strong sensitivitities on \quickmath{\delta_{CP}}, \quickmath{\sin^{2}(\theta_{23})} and \quickmath{\Delta m^{2}_{32}}. To do this, a Bayesian Markov Chain Monte Carlo technique is utilised to generate parameter estimates and credible intervals. Constraints from the T2K near detector are also used to constrain the uncertainties of both beam and atmospheric predictions.

For a known set of oscillation parameters close to the prefered values from a T2K-only data fit, the sensitivity of the joint analysis to \quickmath{\sin^{2}(\theta_{23})} is increased compared to the beam-only analysis. Furthermore, the sensitivity of the joint analysis to select the correct mass hierarchy hypothesis is drastically increased compared to the beam-only analysis, culminating in a substantial preference as classified by Jeffrey's scale. This statement is stronger than the sensitivity of the beam analysis, either with or without external constraints on \quickmath{\sin^{2}(\theta_{13})}. The sensitivities of the beam-only and joint beam-atmospheric analyses have also been compared for a known set of oscillation parameters which are CP-conserving. The joint analysis displays an improved ability to select the correct phase of \quickmath{\delta_{CP}} and octant of \quickmath{\sin^{2}(\theta_{23})} compared to the beam-only analysis. This thesis illustrates the benefit of the combined beam and atmospheric analysis, which could also be extended for use in the next-generation Hyper-Kamiokande experiment.
