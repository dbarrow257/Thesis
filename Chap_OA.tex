\chapter{Oscillation Analysis}
\label{chap:OscillationAnalysis}

\section{Likelihood Calculation}
\label{sec:OscillationAnalysis_LLHCalc}

This analysis performs a joint oscillation parameter fit of the ND280,  and the SK atmospheric samples.

Once the Monte Carlo predictions of each beam and atmospheric sample has been built, following from \autoref{chap:SelsAndSysts}, a likelihood needs to be constructed. This is done by comparing the Monte Carlo prediction to ``data''. The data can consist of either an Asmiov Monte Carlo prediction, which is typically used for sensitivity studies, or real data. The Monte Carlo prediction is calculated at a particular point, \quickmath{\vec{\theta}}, in the model parameter space, \quickmath{N_{i}^{MC} = N_{i}^{MC}(\vec{\theta})}. Both the data and Monte Carlo spectra are binned, where the \quickmath{i^{th}} bin content is represented by \quickmath{N_{i}^{D}} and \quickmath{N_{i}^{MC}}, respectively. The bin contents for the beam near detector, beam far detector and atmospheric samples are denoted with \quickmath{ND}, \quickmath{FD} and \quickmath{Atm}, respectively. The binning index, \quickmath{i}, runs over all the bins within the sample and all samples with that set. Taking the beam far detector samples as example, it would run over all the reconstructed neutrino energy bins in all samples (FHC\quickmath{1\text{R}\mu}, RHC\quickmath{1\text{R}\mu}, etc.). The likelihood calculation between data and Monte Carlo for a particular bin follows a Poisson distribution, where the data is treated as a fluctuation of the simulation. 

Following the T2K analysis presented in \cite{Dunne2020-uf}, the likelihood contribution from the near detector also includes a Monte Carlo statistical uncertainty term, derived from the Barlow and Beeston statistical treatment \cite{Barlow1993-cc, Conway2011-go}. In addition to treating the data as a fluctuation of the Monte Carlo prediction, it includes a contribution from the likelihood that the generated simulation is a statistical fluctuation of the actual true simulation assuming infinite statistics. The technical implementation of this additional likelihood term is documented in \cite{t2k_tn_395}. The term is defined as,

\begin{equation}
  \frac{(\beta_{i}-1)^{2}}{2\sigma^{2}_{\beta_{i}}},
\end{equation}

where \quickmath{\beta_{i}} represents a scaling parameter for each bin \quickmath{i}, which is a value based on the amount of Monte Carlo statistics in a bin \cite{t2k_tn_395}. \quickmath{\sigma_{\beta_{i}} = \sqrt{\sum_{i} w_{i}^{2}}/N_{i}^{MC}}, and \quickmath{\sqrt{\sum_{i} w_{i}^{2}}} represents the sum of the square of the weights of the Monte Carlo events which fall into bin \quickmath{i}.

Additional contributions to the likelihood come from the variation of the systematic model parameters. For those parameters with well-motivated uncertainty estimates, a covariance matrix, \quickmath{V} describes the prior knowledge of each parameter as well as any correlations between the parameters. Due to the technical implementation, a single covariance matrix describes each ``block'' of model parameters, e.g. beam flux systematics. For simplicity, the covariance matrix associated with the \quickmath{k^{th}} block is denoted \quickmath{V^{k}}. This substitution results in \quickmath{\vec{\theta} = \sum_{k}^{N_{b}} \vec{\theta}^{k}} and \quickmath{V = \sum_{k}^{N_{b}} V^{k}}, for \quickmath{N_{b}} number of blocks describing: oscillation parameters, beam flux, atmospheric flux, neutrino interaction, near detector, beam far detector and atmospheric far detector systematics detailed in \autoref{sec:SelsAndSysts_Systs}. The number of parameters in the \quickmath{k^{th}} block is defined as \quickmath{n(k)}.

The final likelihood term is defined as,

\begin{align}
\label{eqn:Likelihood:Likelihood}
&-\ln(\mathcal{L}) = \\ 
& \sum_{i}^{\mathsf{ND bins}} N_{i}^{\mathrm{ND},MC}(\vec{\theta}) - N_{i}^{\mathrm{ND},D} + N_{i}^{\mathrm{ND},D}  \times \ln \left[ N_{i}^{\mathrm{ND},D}/N_{i}^{\mathrm{ND},MC}(\vec{\theta}) \right] + \frac{(\beta_{i}-1)^{2}}{2\sigma^{2}_{\beta_{i}}} \nonumber \\
& +  \sum_{i}^{\mathsf{FD bins}} N_{i}^{\mathrm{FD},MC}(\vec{\theta}) - N_{i}^{\mathrm{FD},D} + N_{i}^{\mathrm{FD},D}  \times \ln \left[ N_{i}^{\mathrm{FD},D}/N_{i}^{\mathrm{FD},MC}(\vec{\theta}) \right] \nonumber \\ 
& +  \sum_{i}^{\mathsf{Atm bins}} N_{i}^{\mathrm{Atm},MC}( \vec{\theta}) - N_{i}^{\mathrm{Atm},D} + N_{i}^{\mathrm{Atm},D} \times  \ln \left[ N_{i}^{\mathrm{Atm},D}/N_{i}^{\mathrm{Atm},MC}(\vec{\theta}) \right] \nonumber \\ 
& + \frac{1}{2} \sum_{k}^{N_{b}} \sum_{i}^{n(k)} \sum_{j}^{n(k)} (\vec{\theta}^{k})_{i} (V^{k})^{-1}_{ij} (\vec{\theta}^{k})_{j}. \nonumber
\end{align}

This is the value determined at each step of the MCMC to build the posterior distribution, as discussed in \autoref{chap:MarkovChainMonteCarlo}.

\iffalse
\subsection{Cross-Fitter Validation}
\label{sec:OscillationAnalysis_CrossFitter}

Alongside the analysis presented within this analysis, an alternative fitter \texttt{P-Theta}, has also been developing the analysis. For the purposes of this analysis, the main benefit of the alternative fitter is to validate the response to each parameter. 
\fi

\subsection{Likelihood Scans}
\label{sec:OscillationAnalysis_LLHScans}

